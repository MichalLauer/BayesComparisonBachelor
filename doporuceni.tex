\begin{quote}
\bfseries\itshape
Tato část šablony nepatří standardně do bakalářské/diplomové práce. Pro finální 
text je zapotřebí:
\begin{itemize}
\item smazat v souboru \verb|prace.tex| řádek \verb|\chapter*{Doporučení pro tvorbu závěrečných prací na FIS}

\begin{quote}
\bfseries\itshape
Tato část šablony nepatří standardně do bakalářské/diplomové práce. Pro finální 
text je zapotřebí:
\begin{itemize}
\item smazat v souboru \verb|prace.tex| řádek \verb|\chapter*{Doporučení pro tvorbu závěrečných prací na FIS}

\begin{quote}
\bfseries\itshape
Tato část šablony nepatří standardně do bakalářské/diplomové práce. Pro finální 
text je zapotřebí:
\begin{itemize}
\item smazat v souboru \verb|prace.tex| řádek \verb|\chapter*{Doporučení pro tvorbu závěrečných prací na FIS}

\begin{quote}
\bfseries\itshape
Tato část šablony nepatří standardně do bakalářské/diplomové práce. Pro finální 
text je zapotřebí:
\begin{itemize}
\item smazat v souboru \verb|prace.tex| řádek \verb|\include{doporuceni}|
\item  a případně též pak zbytečný soubor \verb|doporuceni.tex|.
\end{itemize}
\end{quote}

Následující doporučení pro tvorbu závěrečných prací (dále jen „doporučení“) jsou 
určena k hodnocení obhajitelnosti závěrečné práce. Jsou určena pro všechny typy 
závěrečných prací na všech bakalářských a magisterských studijních programech. 
Nenahrazují posudky k závěrečné práci. Pokud komise považuje závěrečnou práci za 
neobhajitelnou, měla by argumentovat nesplněním některé položky z těchto 
doporučení.

{\bfseries\sffamily\Large Odvedená práce}
\begin{itemize}
\item \vspace*{-2ex}Student provedl odbornou práci v oblasti studijního programu, který studuje (včetně interdisciplinárních oblastí).
\item Závěrečná práce prokazuje studentovu orientaci ve zvolené oblasti a jeho schopnost v této oblasti definovat a splnit zvolený cíl. 
\item Je zřejmé, že student odvedl odbornou práci o pracnosti v rozsahu měsíců.
\end{itemize}

{\bfseries\sffamily\Large Cíle a kontext}
\begin{itemize}
\item \vspace*{-2ex}V textu závěrečné práce jsou popsána východiska – odborný kontext, ze kterého student vychází – situace v odborném poznání nebo situace v konkrétním aplikačním případě.
\item Východiska obsahují jen poznatky, které mají vliv na výsledky závěrečné práce.
\item V textu závěrečné práce je zřetelně popsán cíl, kterého má závěrečná práce dosáhnout; pokud je cílem řešení problému, je tento problém dostatečně vymezen.
\item Je argumentována smysluplnost cílů ve vztahu k východiskům.
\item V textu je argumentována specifičnost hlavního cíle – nejde o generický, mnohokrát zcela stejně řešený problém.
\item Formulace cíle se vztahuje k nějakému odbornému problému, nikoli k textu závěrečné práce samotné, ke čtenáři ani k autorovi. Tedy cíl není formulován jako „napsat text“, „sdělit čtenáři“, „popsat problematiku“, „vysvětlit“, „seznámit se s literaturou z oblasti” apod.
\item Text závěrečné práce je odborný, nikoli populárně naučný. Řeší odborný (praktický nebo teoretický) problém.
\end{itemize}

{\bfseries\sffamily\Large Metodika}
\begin{itemize}
\item \vspace*{-2ex}Text závěrečné práce popisuje postup, podle kterého student pracoval, a to odděleně od výsledků.
\item Postup je popsán v krocích, ze kterých lze odhadnout jejich pracnost.
\item Postup uvádí veškerou práci, kterou student provedl. Pokud jsou v postupu uvedeny kroky, které student neprovedl, jsou tak zřetelně označeny (a je uveden důvod).
\item Postup je popsán tak, že pokud by podle něj postupoval někdo jiný, došel by k obdobným výsledkům jako student.
\item Postup je popsán konkrétně, nikoli jen pomocí obecných názvů myšlenkových postupů typu analýza, dedukce, syntéza apod.
\item Pokud je použit postup podle zavedených a v literatuře popsaných metod, není nutné vysvětlovat detailně jejich fungování. Student by měl ale zdůvodnit svou volbu použitých metod, případně popsat odchylky skutečného postupu oproti zavedeným metodám.
\end{itemize}

{\bfseries\sffamily\Large Výsledky}
\begin{itemize}
\item \vspace*{-2ex}Výsledky prokazují, že student provedl odbornou práci v oblasti studijního programu, který studuje (včetně interdisciplinárních oblastí).
\item Z formulace textu závěrečné práce je zřejmé, co je původním výsledkem studenta, co faktem přebíraným ze zdrojů a co spekulací, resp. diskusí výsledků.
\item Text popisuje a interpretuje výsledky v souladu s postupem.
\item V textu jsou popsány dílčí výsledky jako výstupy jednotlivých kroků. Méně důležité výsledky jsou uvedeny v přílohách, takže text zůstává přehledný.
\item Je doloženo provedení jednotlivých kroků (např. výpočty, popisná statistika, záznamy z rozhovorů, programový kód, deník výzkumníka apod.), např. nahráním příloh do InSIS.
\item Text závěrečné práce popisuje odborné výsledky v logicky navazujícím argumentačním toku. 
\end{itemize}

{\bfseries\sffamily\Large Závěry}
\begin{itemize}
\item \vspace*{-2ex}Závěry hodnotí míru naplnění cíle.
\item Závěry argumentují, jak výsledky přispěly k vyřešení problému.
\item Závěry popisují, jaký je možný vliv výsledků na kontext (situaci v odborném prostředí nebo v konkrétním aplikačním případě), např. možné další pokračování práce.
\item Závěry zmiňují možná omezení dosažených výsledků.
\end{itemize}

{\bfseries\sffamily\Large Originalita}
\begin{itemize}
\item \vspace*{-2ex}Veškeré převzaté, přeložené nebo parafrázované texty jsou řádně označeny a citovány v souladu s citační normou APA 7 (doporučujeme využívat citační nástroj Zotero).
\item V případě použití nástrojů pro automatické generování textů je toto použití v souladu s pravidly a metodickými doporučeními na VŠE.
\item Text závěrečné práce cituje a parafrázuje pouze zdroje, které byly použity pro řešení problému, nebo vymezení kontextu.
\item Text závěrečné práce zbytečně nerekapituluje zřejmé teoretické poznatky (např. ze základních kurzů studijního programu).
\item Pokud student ve výjimečných případech nepracoval zcela sám, jsou u jednotlivých kroků postupu nebo formou tabulky v příloze uvedeni spolupracovníci (firemní, akademičtí) a podíl studenta na jejich provedení.
\end{itemize}

{\bfseries\sffamily\Large Forma}
\begin{itemize}
\item \vspace*{-2ex}Text závěrečné práce je psán jako ucelený souvislý strukturovaný text, jako odstavce členěné do kapitol, ve struktuře vhodné pro řešený problém.
\item Stránky, tabulky, obrázky, přílohy (apod.) jsou číslovány.
\item V závěrečné práci se nevyskytují tabulky, obrázky, přílohy, programový kód (apod.), které nejsou odkazovány ze souvislého textu.
\item Forma závěrečné práce je v souladu s doporučeními zpřístupněnými na intranetu pro studenty FIS.
\item Závěrečná práce může mít formu vědeckého článku. V takovém případě může být doplněna vysvětlujícím úvodem (např. popis časopisu, průběh recenzního řízení, spoluautorství vedoucího závěrečné práce apod.).
\end{itemize}

{\bfseries\sffamily\Large Doplňující nároky pro diplomovou práci}
\begin{itemize}
\item \vspace*{-2ex}Diplomová práce významně prohlubuje oblast poznání v daném tématu.
\item V diplomové práci je jasně specifikován vlastní přínos autora, který je v souladu s cíli práce. 
\item Je nutné provést validaci výsledků diplomové práce (např. srovnání dosažených výsledků s literaturou, matematický důkaz, strukturované rozhovory se zájmovými skupinami, exaktní testování/měření výsledků apod.).
\end{itemize}

{\bfseries\sffamily\Large Specifika týmových závěrečných prací}
\begin{itemize}
\item \vspace*{-2ex}Skutečnost, že bude závěrečná práce realizována v týmu, musí být uvedena v Zadání závěrečné práce uloženém v InSISu, a tedy schválena vedoucím práce a garantem studijního programu (specializace).
\item Každý student z týmu odevzdává individuální závěrečnou práci, která je individuálně posuzována, individuálně obhajována a hodnocena. Každý student je odpovědný za celý text závěrečné práce.
\item Pouze malá část závěrečné práce může být ve vedoucím závěrečné práce schválených případech společná. Více než 70 % závěrečné práce je individuální.
\item V týmu vytvářené artefakty by měly být publikovány například v Git nebo na wiki projektu a autoři závěrečných prací na ně odkazují.
\item Každá v týmu realizovaná závěrečná práce obsahuje přílohu s názvem Podíl členů týmu na výsledku.
\end{itemize}
|
\item  a případně též pak zbytečný soubor \verb|doporuceni.tex|.
\end{itemize}
\end{quote}

Následující doporučení pro tvorbu závěrečných prací (dále jen „doporučení“) jsou 
určena k hodnocení obhajitelnosti závěrečné práce. Jsou určena pro všechny typy 
závěrečných prací na všech bakalářských a magisterských studijních programech. 
Nenahrazují posudky k závěrečné práci. Pokud komise považuje závěrečnou práci za 
neobhajitelnou, měla by argumentovat nesplněním některé položky z těchto 
doporučení.

{\bfseries\sffamily\Large Odvedená práce}
\begin{itemize}
\item \vspace*{-2ex}Student provedl odbornou práci v oblasti studijního programu, který studuje (včetně interdisciplinárních oblastí).
\item Závěrečná práce prokazuje studentovu orientaci ve zvolené oblasti a jeho schopnost v této oblasti definovat a splnit zvolený cíl. 
\item Je zřejmé, že student odvedl odbornou práci o pracnosti v rozsahu měsíců.
\end{itemize}

{\bfseries\sffamily\Large Cíle a kontext}
\begin{itemize}
\item \vspace*{-2ex}V textu závěrečné práce jsou popsána východiska – odborný kontext, ze kterého student vychází – situace v odborném poznání nebo situace v konkrétním aplikačním případě.
\item Východiska obsahují jen poznatky, které mají vliv na výsledky závěrečné práce.
\item V textu závěrečné práce je zřetelně popsán cíl, kterého má závěrečná práce dosáhnout; pokud je cílem řešení problému, je tento problém dostatečně vymezen.
\item Je argumentována smysluplnost cílů ve vztahu k východiskům.
\item V textu je argumentována specifičnost hlavního cíle – nejde o generický, mnohokrát zcela stejně řešený problém.
\item Formulace cíle se vztahuje k nějakému odbornému problému, nikoli k textu závěrečné práce samotné, ke čtenáři ani k autorovi. Tedy cíl není formulován jako „napsat text“, „sdělit čtenáři“, „popsat problematiku“, „vysvětlit“, „seznámit se s literaturou z oblasti” apod.
\item Text závěrečné práce je odborný, nikoli populárně naučný. Řeší odborný (praktický nebo teoretický) problém.
\end{itemize}

{\bfseries\sffamily\Large Metodika}
\begin{itemize}
\item \vspace*{-2ex}Text závěrečné práce popisuje postup, podle kterého student pracoval, a to odděleně od výsledků.
\item Postup je popsán v krocích, ze kterých lze odhadnout jejich pracnost.
\item Postup uvádí veškerou práci, kterou student provedl. Pokud jsou v postupu uvedeny kroky, které student neprovedl, jsou tak zřetelně označeny (a je uveden důvod).
\item Postup je popsán tak, že pokud by podle něj postupoval někdo jiný, došel by k obdobným výsledkům jako student.
\item Postup je popsán konkrétně, nikoli jen pomocí obecných názvů myšlenkových postupů typu analýza, dedukce, syntéza apod.
\item Pokud je použit postup podle zavedených a v literatuře popsaných metod, není nutné vysvětlovat detailně jejich fungování. Student by měl ale zdůvodnit svou volbu použitých metod, případně popsat odchylky skutečného postupu oproti zavedeným metodám.
\end{itemize}

{\bfseries\sffamily\Large Výsledky}
\begin{itemize}
\item \vspace*{-2ex}Výsledky prokazují, že student provedl odbornou práci v oblasti studijního programu, který studuje (včetně interdisciplinárních oblastí).
\item Z formulace textu závěrečné práce je zřejmé, co je původním výsledkem studenta, co faktem přebíraným ze zdrojů a co spekulací, resp. diskusí výsledků.
\item Text popisuje a interpretuje výsledky v souladu s postupem.
\item V textu jsou popsány dílčí výsledky jako výstupy jednotlivých kroků. Méně důležité výsledky jsou uvedeny v přílohách, takže text zůstává přehledný.
\item Je doloženo provedení jednotlivých kroků (např. výpočty, popisná statistika, záznamy z rozhovorů, programový kód, deník výzkumníka apod.), např. nahráním příloh do InSIS.
\item Text závěrečné práce popisuje odborné výsledky v logicky navazujícím argumentačním toku. 
\end{itemize}

{\bfseries\sffamily\Large Závěry}
\begin{itemize}
\item \vspace*{-2ex}Závěry hodnotí míru naplnění cíle.
\item Závěry argumentují, jak výsledky přispěly k vyřešení problému.
\item Závěry popisují, jaký je možný vliv výsledků na kontext (situaci v odborném prostředí nebo v konkrétním aplikačním případě), např. možné další pokračování práce.
\item Závěry zmiňují možná omezení dosažených výsledků.
\end{itemize}

{\bfseries\sffamily\Large Originalita}
\begin{itemize}
\item \vspace*{-2ex}Veškeré převzaté, přeložené nebo parafrázované texty jsou řádně označeny a citovány v souladu s citační normou APA 7 (doporučujeme využívat citační nástroj Zotero).
\item V případě použití nástrojů pro automatické generování textů je toto použití v souladu s pravidly a metodickými doporučeními na VŠE.
\item Text závěrečné práce cituje a parafrázuje pouze zdroje, které byly použity pro řešení problému, nebo vymezení kontextu.
\item Text závěrečné práce zbytečně nerekapituluje zřejmé teoretické poznatky (např. ze základních kurzů studijního programu).
\item Pokud student ve výjimečných případech nepracoval zcela sám, jsou u jednotlivých kroků postupu nebo formou tabulky v příloze uvedeni spolupracovníci (firemní, akademičtí) a podíl studenta na jejich provedení.
\end{itemize}

{\bfseries\sffamily\Large Forma}
\begin{itemize}
\item \vspace*{-2ex}Text závěrečné práce je psán jako ucelený souvislý strukturovaný text, jako odstavce členěné do kapitol, ve struktuře vhodné pro řešený problém.
\item Stránky, tabulky, obrázky, přílohy (apod.) jsou číslovány.
\item V závěrečné práci se nevyskytují tabulky, obrázky, přílohy, programový kód (apod.), které nejsou odkazovány ze souvislého textu.
\item Forma závěrečné práce je v souladu s doporučeními zpřístupněnými na intranetu pro studenty FIS.
\item Závěrečná práce může mít formu vědeckého článku. V takovém případě může být doplněna vysvětlujícím úvodem (např. popis časopisu, průběh recenzního řízení, spoluautorství vedoucího závěrečné práce apod.).
\end{itemize}

{\bfseries\sffamily\Large Doplňující nároky pro diplomovou práci}
\begin{itemize}
\item \vspace*{-2ex}Diplomová práce významně prohlubuje oblast poznání v daném tématu.
\item V diplomové práci je jasně specifikován vlastní přínos autora, který je v souladu s cíli práce. 
\item Je nutné provést validaci výsledků diplomové práce (např. srovnání dosažených výsledků s literaturou, matematický důkaz, strukturované rozhovory se zájmovými skupinami, exaktní testování/měření výsledků apod.).
\end{itemize}

{\bfseries\sffamily\Large Specifika týmových závěrečných prací}
\begin{itemize}
\item \vspace*{-2ex}Skutečnost, že bude závěrečná práce realizována v týmu, musí být uvedena v Zadání závěrečné práce uloženém v InSISu, a tedy schválena vedoucím práce a garantem studijního programu (specializace).
\item Každý student z týmu odevzdává individuální závěrečnou práci, která je individuálně posuzována, individuálně obhajována a hodnocena. Každý student je odpovědný za celý text závěrečné práce.
\item Pouze malá část závěrečné práce může být ve vedoucím závěrečné práce schválených případech společná. Více než 70 % závěrečné práce je individuální.
\item V týmu vytvářené artefakty by měly být publikovány například v Git nebo na wiki projektu a autoři závěrečných prací na ně odkazují.
\item Každá v týmu realizovaná závěrečná práce obsahuje přílohu s názvem Podíl členů týmu na výsledku.
\end{itemize}
|
\item  a případně též pak zbytečný soubor \verb|doporuceni.tex|.
\end{itemize}
\end{quote}

Následující doporučení pro tvorbu závěrečných prací (dále jen „doporučení“) jsou 
určena k hodnocení obhajitelnosti závěrečné práce. Jsou určena pro všechny typy 
závěrečných prací na všech bakalářských a magisterských studijních programech. 
Nenahrazují posudky k závěrečné práci. Pokud komise považuje závěrečnou práci za 
neobhajitelnou, měla by argumentovat nesplněním některé položky z těchto 
doporučení.

{\bfseries\sffamily\Large Odvedená práce}
\begin{itemize}
\item \vspace*{-2ex}Student provedl odbornou práci v oblasti studijního programu, který studuje (včetně interdisciplinárních oblastí).
\item Závěrečná práce prokazuje studentovu orientaci ve zvolené oblasti a jeho schopnost v této oblasti definovat a splnit zvolený cíl. 
\item Je zřejmé, že student odvedl odbornou práci o pracnosti v rozsahu měsíců.
\end{itemize}

{\bfseries\sffamily\Large Cíle a kontext}
\begin{itemize}
\item \vspace*{-2ex}V textu závěrečné práce jsou popsána východiska – odborný kontext, ze kterého student vychází – situace v odborném poznání nebo situace v konkrétním aplikačním případě.
\item Východiska obsahují jen poznatky, které mají vliv na výsledky závěrečné práce.
\item V textu závěrečné práce je zřetelně popsán cíl, kterého má závěrečná práce dosáhnout; pokud je cílem řešení problému, je tento problém dostatečně vymezen.
\item Je argumentována smysluplnost cílů ve vztahu k východiskům.
\item V textu je argumentována specifičnost hlavního cíle – nejde o generický, mnohokrát zcela stejně řešený problém.
\item Formulace cíle se vztahuje k nějakému odbornému problému, nikoli k textu závěrečné práce samotné, ke čtenáři ani k autorovi. Tedy cíl není formulován jako „napsat text“, „sdělit čtenáři“, „popsat problematiku“, „vysvětlit“, „seznámit se s literaturou z oblasti” apod.
\item Text závěrečné práce je odborný, nikoli populárně naučný. Řeší odborný (praktický nebo teoretický) problém.
\end{itemize}

{\bfseries\sffamily\Large Metodika}
\begin{itemize}
\item \vspace*{-2ex}Text závěrečné práce popisuje postup, podle kterého student pracoval, a to odděleně od výsledků.
\item Postup je popsán v krocích, ze kterých lze odhadnout jejich pracnost.
\item Postup uvádí veškerou práci, kterou student provedl. Pokud jsou v postupu uvedeny kroky, které student neprovedl, jsou tak zřetelně označeny (a je uveden důvod).
\item Postup je popsán tak, že pokud by podle něj postupoval někdo jiný, došel by k obdobným výsledkům jako student.
\item Postup je popsán konkrétně, nikoli jen pomocí obecných názvů myšlenkových postupů typu analýza, dedukce, syntéza apod.
\item Pokud je použit postup podle zavedených a v literatuře popsaných metod, není nutné vysvětlovat detailně jejich fungování. Student by měl ale zdůvodnit svou volbu použitých metod, případně popsat odchylky skutečného postupu oproti zavedeným metodám.
\end{itemize}

{\bfseries\sffamily\Large Výsledky}
\begin{itemize}
\item \vspace*{-2ex}Výsledky prokazují, že student provedl odbornou práci v oblasti studijního programu, který studuje (včetně interdisciplinárních oblastí).
\item Z formulace textu závěrečné práce je zřejmé, co je původním výsledkem studenta, co faktem přebíraným ze zdrojů a co spekulací, resp. diskusí výsledků.
\item Text popisuje a interpretuje výsledky v souladu s postupem.
\item V textu jsou popsány dílčí výsledky jako výstupy jednotlivých kroků. Méně důležité výsledky jsou uvedeny v přílohách, takže text zůstává přehledný.
\item Je doloženo provedení jednotlivých kroků (např. výpočty, popisná statistika, záznamy z rozhovorů, programový kód, deník výzkumníka apod.), např. nahráním příloh do InSIS.
\item Text závěrečné práce popisuje odborné výsledky v logicky navazujícím argumentačním toku. 
\end{itemize}

{\bfseries\sffamily\Large Závěry}
\begin{itemize}
\item \vspace*{-2ex}Závěry hodnotí míru naplnění cíle.
\item Závěry argumentují, jak výsledky přispěly k vyřešení problému.
\item Závěry popisují, jaký je možný vliv výsledků na kontext (situaci v odborném prostředí nebo v konkrétním aplikačním případě), např. možné další pokračování práce.
\item Závěry zmiňují možná omezení dosažených výsledků.
\end{itemize}

{\bfseries\sffamily\Large Originalita}
\begin{itemize}
\item \vspace*{-2ex}Veškeré převzaté, přeložené nebo parafrázované texty jsou řádně označeny a citovány v souladu s citační normou APA 7 (doporučujeme využívat citační nástroj Zotero).
\item V případě použití nástrojů pro automatické generování textů je toto použití v souladu s pravidly a metodickými doporučeními na VŠE.
\item Text závěrečné práce cituje a parafrázuje pouze zdroje, které byly použity pro řešení problému, nebo vymezení kontextu.
\item Text závěrečné práce zbytečně nerekapituluje zřejmé teoretické poznatky (např. ze základních kurzů studijního programu).
\item Pokud student ve výjimečných případech nepracoval zcela sám, jsou u jednotlivých kroků postupu nebo formou tabulky v příloze uvedeni spolupracovníci (firemní, akademičtí) a podíl studenta na jejich provedení.
\end{itemize}

{\bfseries\sffamily\Large Forma}
\begin{itemize}
\item \vspace*{-2ex}Text závěrečné práce je psán jako ucelený souvislý strukturovaný text, jako odstavce členěné do kapitol, ve struktuře vhodné pro řešený problém.
\item Stránky, tabulky, obrázky, přílohy (apod.) jsou číslovány.
\item V závěrečné práci se nevyskytují tabulky, obrázky, přílohy, programový kód (apod.), které nejsou odkazovány ze souvislého textu.
\item Forma závěrečné práce je v souladu s doporučeními zpřístupněnými na intranetu pro studenty FIS.
\item Závěrečná práce může mít formu vědeckého článku. V takovém případě může být doplněna vysvětlujícím úvodem (např. popis časopisu, průběh recenzního řízení, spoluautorství vedoucího závěrečné práce apod.).
\end{itemize}

{\bfseries\sffamily\Large Doplňující nároky pro diplomovou práci}
\begin{itemize}
\item \vspace*{-2ex}Diplomová práce významně prohlubuje oblast poznání v daném tématu.
\item V diplomové práci je jasně specifikován vlastní přínos autora, který je v souladu s cíli práce. 
\item Je nutné provést validaci výsledků diplomové práce (např. srovnání dosažených výsledků s literaturou, matematický důkaz, strukturované rozhovory se zájmovými skupinami, exaktní testování/měření výsledků apod.).
\end{itemize}

{\bfseries\sffamily\Large Specifika týmových závěrečných prací}
\begin{itemize}
\item \vspace*{-2ex}Skutečnost, že bude závěrečná práce realizována v týmu, musí být uvedena v Zadání závěrečné práce uloženém v InSISu, a tedy schválena vedoucím práce a garantem studijního programu (specializace).
\item Každý student z týmu odevzdává individuální závěrečnou práci, která je individuálně posuzována, individuálně obhajována a hodnocena. Každý student je odpovědný za celý text závěrečné práce.
\item Pouze malá část závěrečné práce může být ve vedoucím závěrečné práce schválených případech společná. Více než 70 % závěrečné práce je individuální.
\item V týmu vytvářené artefakty by měly být publikovány například v Git nebo na wiki projektu a autoři závěrečných prací na ně odkazují.
\item Každá v týmu realizovaná závěrečná práce obsahuje přílohu s názvem Podíl členů týmu na výsledku.
\end{itemize}
|
\item  a případně též pak zbytečný soubor \verb|doporuceni.tex|.
\end{itemize}
\end{quote}

Následující doporučení pro tvorbu závěrečných prací (dále jen „doporučení“) jsou 
určena k hodnocení obhajitelnosti závěrečné práce. Jsou určena pro všechny typy 
závěrečných prací na všech bakalářských a magisterských studijních programech. 
Nenahrazují posudky k závěrečné práci. Pokud komise považuje závěrečnou práci za 
neobhajitelnou, měla by argumentovat nesplněním některé položky z těchto 
doporučení.

{\bfseries\sffamily\Large Odvedená práce}
\begin{itemize}
\item \vspace*{-2ex}Student provedl odbornou práci v oblasti studijního programu, který studuje (včetně interdisciplinárních oblastí).
\item Závěrečná práce prokazuje studentovu orientaci ve zvolené oblasti a jeho schopnost v této oblasti definovat a splnit zvolený cíl. 
\item Je zřejmé, že student odvedl odbornou práci o pracnosti v rozsahu měsíců.
\end{itemize}

{\bfseries\sffamily\Large Cíle a kontext}
\begin{itemize}
\item \vspace*{-2ex}V textu závěrečné práce jsou popsána východiska – odborný kontext, ze kterého student vychází – situace v odborném poznání nebo situace v konkrétním aplikačním případě.
\item Východiska obsahují jen poznatky, které mají vliv na výsledky závěrečné práce.
\item V textu závěrečné práce je zřetelně popsán cíl, kterého má závěrečná práce dosáhnout; pokud je cílem řešení problému, je tento problém dostatečně vymezen.
\item Je argumentována smysluplnost cílů ve vztahu k východiskům.
\item V textu je argumentována specifičnost hlavního cíle – nejde o generický, mnohokrát zcela stejně řešený problém.
\item Formulace cíle se vztahuje k nějakému odbornému problému, nikoli k textu závěrečné práce samotné, ke čtenáři ani k autorovi. Tedy cíl není formulován jako „napsat text“, „sdělit čtenáři“, „popsat problematiku“, „vysvětlit“, „seznámit se s literaturou z oblasti” apod.
\item Text závěrečné práce je odborný, nikoli populárně naučný. Řeší odborný (praktický nebo teoretický) problém.
\end{itemize}

{\bfseries\sffamily\Large Metodika}
\begin{itemize}
\item \vspace*{-2ex}Text závěrečné práce popisuje postup, podle kterého student pracoval, a to odděleně od výsledků.
\item Postup je popsán v krocích, ze kterých lze odhadnout jejich pracnost.
\item Postup uvádí veškerou práci, kterou student provedl. Pokud jsou v postupu uvedeny kroky, které student neprovedl, jsou tak zřetelně označeny (a je uveden důvod).
\item Postup je popsán tak, že pokud by podle něj postupoval někdo jiný, došel by k obdobným výsledkům jako student.
\item Postup je popsán konkrétně, nikoli jen pomocí obecných názvů myšlenkových postupů typu analýza, dedukce, syntéza apod.
\item Pokud je použit postup podle zavedených a v literatuře popsaných metod, není nutné vysvětlovat detailně jejich fungování. Student by měl ale zdůvodnit svou volbu použitých metod, případně popsat odchylky skutečného postupu oproti zavedeným metodám.
\end{itemize}

{\bfseries\sffamily\Large Výsledky}
\begin{itemize}
\item \vspace*{-2ex}Výsledky prokazují, že student provedl odbornou práci v oblasti studijního programu, který studuje (včetně interdisciplinárních oblastí).
\item Z formulace textu závěrečné práce je zřejmé, co je původním výsledkem studenta, co faktem přebíraným ze zdrojů a co spekulací, resp. diskusí výsledků.
\item Text popisuje a interpretuje výsledky v souladu s postupem.
\item V textu jsou popsány dílčí výsledky jako výstupy jednotlivých kroků. Méně důležité výsledky jsou uvedeny v přílohách, takže text zůstává přehledný.
\item Je doloženo provedení jednotlivých kroků (např. výpočty, popisná statistika, záznamy z rozhovorů, programový kód, deník výzkumníka apod.), např. nahráním příloh do InSIS.
\item Text závěrečné práce popisuje odborné výsledky v logicky navazujícím argumentačním toku. 
\end{itemize}

{\bfseries\sffamily\Large Závěry}
\begin{itemize}
\item \vspace*{-2ex}Závěry hodnotí míru naplnění cíle.
\item Závěry argumentují, jak výsledky přispěly k vyřešení problému.
\item Závěry popisují, jaký je možný vliv výsledků na kontext (situaci v odborném prostředí nebo v konkrétním aplikačním případě), např. možné další pokračování práce.
\item Závěry zmiňují možná omezení dosažených výsledků.
\end{itemize}

{\bfseries\sffamily\Large Originalita}
\begin{itemize}
\item \vspace*{-2ex}Veškeré převzaté, přeložené nebo parafrázované texty jsou řádně označeny a citovány v souladu s citační normou APA 7 (doporučujeme využívat citační nástroj Zotero).
\item V případě použití nástrojů pro automatické generování textů je toto použití v souladu s pravidly a metodickými doporučeními na VŠE.
\item Text závěrečné práce cituje a parafrázuje pouze zdroje, které byly použity pro řešení problému, nebo vymezení kontextu.
\item Text závěrečné práce zbytečně nerekapituluje zřejmé teoretické poznatky (např. ze základních kurzů studijního programu).
\item Pokud student ve výjimečných případech nepracoval zcela sám, jsou u jednotlivých kroků postupu nebo formou tabulky v příloze uvedeni spolupracovníci (firemní, akademičtí) a podíl studenta na jejich provedení.
\end{itemize}

{\bfseries\sffamily\Large Forma}
\begin{itemize}
\item \vspace*{-2ex}Text závěrečné práce je psán jako ucelený souvislý strukturovaný text, jako odstavce členěné do kapitol, ve struktuře vhodné pro řešený problém.
\item Stránky, tabulky, obrázky, přílohy (apod.) jsou číslovány.
\item V závěrečné práci se nevyskytují tabulky, obrázky, přílohy, programový kód (apod.), které nejsou odkazovány ze souvislého textu.
\item Forma závěrečné práce je v souladu s doporučeními zpřístupněnými na intranetu pro studenty FIS.
\item Závěrečná práce může mít formu vědeckého článku. V takovém případě může být doplněna vysvětlujícím úvodem (např. popis časopisu, průběh recenzního řízení, spoluautorství vedoucího závěrečné práce apod.).
\end{itemize}

{\bfseries\sffamily\Large Doplňující nároky pro diplomovou práci}
\begin{itemize}
\item \vspace*{-2ex}Diplomová práce významně prohlubuje oblast poznání v daném tématu.
\item V diplomové práci je jasně specifikován vlastní přínos autora, který je v souladu s cíli práce. 
\item Je nutné provést validaci výsledků diplomové práce (např. srovnání dosažených výsledků s literaturou, matematický důkaz, strukturované rozhovory se zájmovými skupinami, exaktní testování/měření výsledků apod.).
\end{itemize}

{\bfseries\sffamily\Large Specifika týmových závěrečných prací}
\begin{itemize}
\item \vspace*{-2ex}Skutečnost, že bude závěrečná práce realizována v týmu, musí být uvedena v Zadání závěrečné práce uloženém v InSISu, a tedy schválena vedoucím práce a garantem studijního programu (specializace).
\item Každý student z týmu odevzdává individuální závěrečnou práci, která je individuálně posuzována, individuálně obhajována a hodnocena. Každý student je odpovědný za celý text závěrečné práce.
\item Pouze malá část závěrečné práce může být ve vedoucím závěrečné práce schválených případech společná. Více než 70 % závěrečné práce je individuální.
\item V týmu vytvářené artefakty by měly být publikovány například v Git nebo na wiki projektu a autoři závěrečných prací na ně odkazují.
\item Každá v týmu realizovaná závěrečná práce obsahuje přílohu s názvem Podíl členů týmu na výsledku.
\end{itemize}
