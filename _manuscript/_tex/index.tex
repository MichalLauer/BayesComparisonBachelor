% Options for packages loaded elsewhere
\PassOptionsToPackage{unicode}{hyperref}
\PassOptionsToPackage{hyphens}{url}
\PassOptionsToPackage{dvipsnames,svgnames,x11names}{xcolor}
%
\documentclass[
  11pt,
  a4paper]{report}

\usepackage{amsmath,amssymb}
\usepackage{iftex}
\ifPDFTeX
  \usepackage[T1]{fontenc}
  \usepackage[utf8]{inputenc}
  \usepackage{textcomp} % provide euro and other symbols
\else % if luatex or xetex
  \usepackage{unicode-math}
  \defaultfontfeatures{Scale=MatchLowercase}
  \defaultfontfeatures[\rmfamily]{Ligatures=TeX,Scale=1}
\fi
\usepackage{lmodern}
\ifPDFTeX\else  
    % xetex/luatex font selection
\fi
% Use upquote if available, for straight quotes in verbatim environments
\IfFileExists{upquote.sty}{\usepackage{upquote}}{}
\IfFileExists{microtype.sty}{% use microtype if available
  \usepackage[]{microtype}
  \UseMicrotypeSet[protrusion]{basicmath} % disable protrusion for tt fonts
}{}
\makeatletter
\@ifundefined{KOMAClassName}{% if non-KOMA class
  \IfFileExists{parskip.sty}{%
    \usepackage{parskip}
  }{% else
    \setlength{\parindent}{0pt}
    \setlength{\parskip}{6pt plus 2pt minus 1pt}}
}{% if KOMA class
  \KOMAoptions{parskip=half}}
\makeatother
\usepackage{xcolor}
\setlength{\emergencystretch}{3em} % prevent overfull lines
\setcounter{secnumdepth}{-\maxdimen} % remove section numbering
% Make \paragraph and \subparagraph free-standing
\ifx\paragraph\undefined\else
  \let\oldparagraph\paragraph
  \renewcommand{\paragraph}[1]{\oldparagraph{#1}\mbox{}}
\fi
\ifx\subparagraph\undefined\else
  \let\oldsubparagraph\subparagraph
  \renewcommand{\subparagraph}[1]{\oldsubparagraph{#1}\mbox{}}
\fi

\usepackage{color}
\usepackage{fancyvrb}
\newcommand{\VerbBar}{|}
\newcommand{\VERB}{\Verb[commandchars=\\\{\}]}
\DefineVerbatimEnvironment{Highlighting}{Verbatim}{commandchars=\\\{\}}
% Add ',fontsize=\small' for more characters per line
\usepackage{framed}
\definecolor{shadecolor}{RGB}{241,243,245}
\newenvironment{Shaded}{\begin{snugshade}}{\end{snugshade}}
\newcommand{\AlertTok}[1]{\textcolor[rgb]{0.68,0.00,0.00}{#1}}
\newcommand{\AnnotationTok}[1]{\textcolor[rgb]{0.37,0.37,0.37}{#1}}
\newcommand{\AttributeTok}[1]{\textcolor[rgb]{0.40,0.45,0.13}{#1}}
\newcommand{\BaseNTok}[1]{\textcolor[rgb]{0.68,0.00,0.00}{#1}}
\newcommand{\BuiltInTok}[1]{\textcolor[rgb]{0.00,0.23,0.31}{#1}}
\newcommand{\CharTok}[1]{\textcolor[rgb]{0.13,0.47,0.30}{#1}}
\newcommand{\CommentTok}[1]{\textcolor[rgb]{0.37,0.37,0.37}{#1}}
\newcommand{\CommentVarTok}[1]{\textcolor[rgb]{0.37,0.37,0.37}{\textit{#1}}}
\newcommand{\ConstantTok}[1]{\textcolor[rgb]{0.56,0.35,0.01}{#1}}
\newcommand{\ControlFlowTok}[1]{\textcolor[rgb]{0.00,0.23,0.31}{#1}}
\newcommand{\DataTypeTok}[1]{\textcolor[rgb]{0.68,0.00,0.00}{#1}}
\newcommand{\DecValTok}[1]{\textcolor[rgb]{0.68,0.00,0.00}{#1}}
\newcommand{\DocumentationTok}[1]{\textcolor[rgb]{0.37,0.37,0.37}{\textit{#1}}}
\newcommand{\ErrorTok}[1]{\textcolor[rgb]{0.68,0.00,0.00}{#1}}
\newcommand{\ExtensionTok}[1]{\textcolor[rgb]{0.00,0.23,0.31}{#1}}
\newcommand{\FloatTok}[1]{\textcolor[rgb]{0.68,0.00,0.00}{#1}}
\newcommand{\FunctionTok}[1]{\textcolor[rgb]{0.28,0.35,0.67}{#1}}
\newcommand{\ImportTok}[1]{\textcolor[rgb]{0.00,0.46,0.62}{#1}}
\newcommand{\InformationTok}[1]{\textcolor[rgb]{0.37,0.37,0.37}{#1}}
\newcommand{\KeywordTok}[1]{\textcolor[rgb]{0.00,0.23,0.31}{#1}}
\newcommand{\NormalTok}[1]{\textcolor[rgb]{0.00,0.23,0.31}{#1}}
\newcommand{\OperatorTok}[1]{\textcolor[rgb]{0.37,0.37,0.37}{#1}}
\newcommand{\OtherTok}[1]{\textcolor[rgb]{0.00,0.23,0.31}{#1}}
\newcommand{\PreprocessorTok}[1]{\textcolor[rgb]{0.68,0.00,0.00}{#1}}
\newcommand{\RegionMarkerTok}[1]{\textcolor[rgb]{0.00,0.23,0.31}{#1}}
\newcommand{\SpecialCharTok}[1]{\textcolor[rgb]{0.37,0.37,0.37}{#1}}
\newcommand{\SpecialStringTok}[1]{\textcolor[rgb]{0.13,0.47,0.30}{#1}}
\newcommand{\StringTok}[1]{\textcolor[rgb]{0.13,0.47,0.30}{#1}}
\newcommand{\VariableTok}[1]{\textcolor[rgb]{0.07,0.07,0.07}{#1}}
\newcommand{\VerbatimStringTok}[1]{\textcolor[rgb]{0.13,0.47,0.30}{#1}}
\newcommand{\WarningTok}[1]{\textcolor[rgb]{0.37,0.37,0.37}{\textit{#1}}}

\providecommand{\tightlist}{%
  \setlength{\itemsep}{0pt}\setlength{\parskip}{0pt}}\usepackage{longtable,booktabs,array}
\usepackage{calc} % for calculating minipage widths
% Correct order of tables after \paragraph or \subparagraph
\usepackage{etoolbox}
\makeatletter
\patchcmd\longtable{\par}{\if@noskipsec\mbox{}\fi\par}{}{}
\makeatother
% Allow footnotes in longtable head/foot
\IfFileExists{footnotehyper.sty}{\usepackage{footnotehyper}}{\usepackage{footnote}}
\makesavenoteenv{longtable}
\usepackage{graphicx}
\makeatletter
\def\maxwidth{\ifdim\Gin@nat@width>\linewidth\linewidth\else\Gin@nat@width\fi}
\def\maxheight{\ifdim\Gin@nat@height>\textheight\textheight\else\Gin@nat@height\fi}
\makeatother
% Scale images if necessary, so that they will not overflow the page
% margins by default, and it is still possible to overwrite the defaults
% using explicit options in \includegraphics[width, height, ...]{}
\setkeys{Gin}{width=\maxwidth,height=\maxheight,keepaspectratio}
% Set default figure placement to htbp
\makeatletter
\def\fps@figure{htbp}
\makeatother

\usepackage[top=25mm,bottom=25mm,right=25mm,left=30mm,head=12.5mm,foot=12.5mm]{geometry}
\let\openright=\clearpage

%%% Pokud tiskneme oboustranně:
%\documentclass[11pt,a4paper,twoside,openright]{report}
%\usepackage[top=25mm,bottom=25mm,right=25mm,left=30mm,head=12.5mm,foot=12.5mm]{geometry}
%\let\openright=\cleardoublepage

%%% DEFINICE ZÁKLADNÍCH PROMĚNNÝCH
\def\TypPrace{BP}                % bakalářská práce/bachelor thesis
%\def\TypPrace{DP}               % diplomová práce/master thesis
\def\Jazyk{cze}                  % čeština/czech

%%% Definice různých užitečných maker (viz popis uvnitř souboru)
\input{setup/makra}

%%% Název práce v jazyce práce (přesně podle zadání)
%%% Title of the thesis in the language used in the text (exact according to assignment)
\def\NazevPrace{Odhad relativní četnosti binomického rozdělení pomocí klasického a bayesovského přístupu v jazyce R}

%%% Jméno autora
%%% Author's name - First name Surname
\def\AutorPrace{[Bc. Michal Lauer]}

%%% Rok odevzdání a měsíc (slovně)
%%% Year of submission and month (verbally) - month YYYY
\def\DatumOdevzdani{Prosinec 2024}

%%% Vedoucí práce: Jméno a příjmení s~tituly
%%% Supervisor: First name and surname with titles
\def\Vedouci{[Ing. Ondřej Vilikus, Ph.D.]}

% %%% Konzultant práce: Jméno a příjmení s~tituly
% %%% Consultant: First name and surname with titles
\def\Konzultant{}

%%% Studijní program
%%% Study program
\def\StudijniProgram{[Data Analytics]}

%%% Studijní program - specializace
%%% Study program - specialization
\def\Specializace{}

%%% Nepovinné poděkování (vedoucímu práce, konzultantovi, tomu, kdo zapůjčil software, literaturu apod.)
%%% Optional thanks (the supervisor, the consultant, the borrower of software, literature, etc.)
\def\Podekovani{%
Děkuji svému vedoucímu za odborné vedení práce a průběžné konzultace a své přítelkyni za neocenitelnou podporu.
}

%%% Abstrakt (doporučený rozsah cca 150-250 slov; nejedná se o zadání práce)
\def\Abstrakt{%
Abstrakt.
}
\def\AbstraktEN{%
Abstract.
}

%%% 3 až 5 klíčových slov (doporučeno)
\def\KlicovaSlova{Bayesovská statistika, odhad relativní četnosti, jazyk R}
\def\KlicovaSlovaEN{Bayesian statistics, relative frequency estimation, R language}

% Quarto fixes
\setcounter{secnumdepth}{2}
\makeatletter
\@ifpackageloaded{caption}{}{\usepackage{caption}}
\AtBeginDocument{%
\ifdefined\contentsname
  \renewcommand*\contentsname{Obsah}
\else
  \newcommand\contentsname{Obsah}
\fi
\ifdefined\listfigurename
  \renewcommand*\listfigurename{Seznam obrázků}
\else
  \newcommand\listfigurename{Seznam obrázků}
\fi
\ifdefined\listtablename
  \renewcommand*\listtablename{Seznam tabulek}
\else
  \newcommand\listtablename{Seznam tabulek}
\fi
\ifdefined\figurename
  \renewcommand*\figurename{Obrázek}
\else
  \newcommand\figurename{Obrázek}
\fi
\ifdefined\tablename
  \renewcommand*\tablename{Tabulka}
\else
  \newcommand\tablename{Tabulka}
\fi
}
\@ifpackageloaded{float}{}{\usepackage{float}}
\floatstyle{ruled}
\@ifundefined{c@chapter}{\newfloat{codelisting}{h}{lop}}{\newfloat{codelisting}{h}{lop}[chapter]}
\floatname{codelisting}{Výpis}
\newcommand*\listoflistings{\listof{codelisting}{Seznam výpisů}}
\makeatother
\makeatletter
\makeatother
\makeatletter
\@ifpackageloaded{caption}{}{\usepackage{caption}}
\@ifpackageloaded{subcaption}{}{\usepackage{subcaption}}
\makeatother
\ifLuaTeX
\usepackage[bidi=basic]{babel}
\else
\usepackage[bidi=default]{babel}
\fi
\babelprovide[main,import]{czech}
% get rid of language-specific shorthands (see #6817):
\let\LanguageShortHands\languageshorthands
\def\languageshorthands#1{}
\ifLuaTeX
  \usepackage{selnolig}  % disable illegal ligatures
\fi
\usepackage{bookmark}

\IfFileExists{xurl.sty}{\usepackage{xurl}}{} % add URL line breaks if available
\urlstyle{same} % disable monospaced font for URLs
\hypersetup{
  pdflang={cs},
  colorlinks=true,
  linkcolor={blue},
  filecolor={Maroon},
  citecolor={Blue},
  urlcolor={Blue},
  pdfcreator={LaTeX via pandoc}}

\author{}
\date{}

\begin{document}

\include{interim/intro.tex}

\include{interim/toc.tex}

\openright
\listoffigures
Poznámka: Seznam obrázků je vhodné použít, pokud počet obrázků v textu práce je větší než 20.

\clearpage
\listoftables
Poznámka: Seznam tabulek je vhodný použít, pokud počet tabulek v textu práce je větší než 20.

\clearpage
\lstlistoflistings
Poznámka: Seznam výpisů programového kódu je vhodné použít, pokud počet vložených objektů tohoto typu je větší než 20. 

\include{interim/zkratky.tex}

\include{uvod.tex}

\chapter{Statistické metody}\label{statistickuxe9-metody}

Krátký úvod do historie, bayes, inferenční bayes (rozdělení)
vs.~inference (bod) citace Karla

\section{Inference}\label{inference}

proč to používáme, výběr vs.~populace, reprezentativnost

\subsection{Problematika výběrových
šetření}\label{problematika-vuxfdbux11brovuxfdch-ux161etux159enuxed}

reprezentativnost, definice populace, čas sběru, organizace
sběru\ldots{}

\section{Frekventistická inference}\label{frekventistickuxe1-inference}

Jak to funguje, jak to spoléhá na sampling distributions

\subsection{Testování hypotéz}\label{testovuxe1nuxed-hypotuxe9z}

hladina významnosti, úroveň spolehlivosti, Testovací statistika,
kritický obor, 1/2 stranný test p-hodnota, interval spolehlivosti

\subsection{Metriky při testování
hypotéz}\label{metriky-pux159i-testovuxe1nuxed-hypotuxe9z}

Chyba I. a II. druhu, síla testu, velikost efektu

\subsection{Jednovýběrový odhad poměru s velkým
vzorkem}\label{jednovuxfdbux11brovuxfd-odhad-pomux11bru-s-velkuxfdm-vzorkem}

použití, předpoklady, poměrový Z test, binomický test, síla testu,
velikost efektu

\subsection{Jednovýběrový odhad poměru s malým
vzorkem}\label{jednovuxfdbux11brovuxfd-odhad-pomux11bru-s-maluxfdm-vzorkem}

Proč jsou důležité speciální metody, nějaké typy (wiki)

\section{Bayesovská inference}\label{bayesovskuxe1-inference}

Odvození bayesova vzorce, popis likelihood/aprior/data, druhy
aprior/posterior

\chapter{Monte Carlo generování}\label{monte-carlo-generovuxe1nuxed}

Halsing, Gibs, HMC

\section{Vyhodnocení generovaného
rozdělení}\label{vyhodnocenuxed-generovanuxe9ho-rozdux11blenuxed}

korelace, ESS, monte carlo error\ldots{}

\subsection{Vyhodocení hypotéz}\label{vyhodocenuxed-hypotuxe9z}

Interval kredibility, ROPE, Bayesův faktor

\subsection{Odhad poměru}\label{odhad-pomux11bru}

\chapter{Praktické odhady}\label{praktickuxe9-odhady}

Test

test

\texttt{stats::t.test()}

\textbf{test}

Jednoduchý T-test

\begin{Shaded}
\begin{Highlighting}[]
\FunctionTok{set.seed}\NormalTok{(}\DecValTok{78}\NormalTok{)}
\NormalTok{x }\OtherTok{\textless{}{-}} \FunctionTok{rnorm}\NormalTok{(}\AttributeTok{n =} \DecValTok{100}\NormalTok{, }\AttributeTok{mean =} \DecValTok{1}\NormalTok{, }\AttributeTok{sd =} \DecValTok{1}\NormalTok{)}
\FunctionTok{t.test}\NormalTok{(}\AttributeTok{x =}\NormalTok{ x, }\AttributeTok{mu =} \DecValTok{0}\NormalTok{,}
       \AttributeTok{alternative =} \StringTok{"two.sided"}\NormalTok{, }\AttributeTok{conf.level =} \FloatTok{0.95}\NormalTok{)}
\end{Highlighting}
\end{Shaded}

\begin{verbatim}

    One Sample t-test

data:  x
t = 8.8438, df = 99, p-value = 3.621e-14
alternative hypothesis: true mean is not equal to 0
95 percent confidence interval:
 0.7158758 1.1300282
sample estimates:
mean of x 
 0.922952 
\end{verbatim}

Simulace alfa = chyba 1. druhu

\begin{Shaded}
\begin{Highlighting}[]
\FunctionTok{library}\NormalTok{(dplyr)}
\end{Highlighting}
\end{Shaded}

\begin{Shaded}
\begin{Highlighting}[]
\FunctionTok{library}\NormalTok{(ggplot2)}
\end{Highlighting}
\end{Shaded}

\begin{Shaded}
\begin{Highlighting}[]
\CommentTok{\# Nastavení}
\NormalTok{K   }\OtherTok{\textless{}{-}} \DecValTok{1000}   \CommentTok{\# Počet simulací}
\NormalTok{n   }\OtherTok{\textless{}{-}} \DecValTok{200}    \CommentTok{\# Velikost vzorku}
\NormalTok{mu0 }\OtherTok{\textless{}{-}} \DecValTok{0}      \CommentTok{\# Skutečný průměr}
\NormalTok{sd0 }\OtherTok{\textless{}{-}} \DecValTok{2}      \CommentTok{\# Populační směrodatná odchylka}
\NormalTok{alpha }\OtherTok{\textless{}{-}} \FloatTok{0.05} \CommentTok{\# Hladina významnosti}
\FunctionTok{set.seed}\NormalTok{(}\DecValTok{639}\NormalTok{)}
\NormalTok{vzorky }\OtherTok{\textless{}{-}} \FunctionTok{tibble}\NormalTok{()}

\CommentTok{\# Simulace}
\ControlFlowTok{for}\NormalTok{ (i }\ControlFlowTok{in} \FunctionTok{seq\_len}\NormalTok{(K)) \{}
\NormalTok{    x }\OtherTok{\textless{}{-}} \FunctionTok{rnorm}\NormalTok{(}\AttributeTok{n =}\NormalTok{ n, }\AttributeTok{mean =}\NormalTok{ mu0, }\AttributeTok{sd =}\NormalTok{ sd0)}
\NormalTok{    test }\OtherTok{\textless{}{-}} \FunctionTok{t.test}\NormalTok{(}\AttributeTok{x =}\NormalTok{ x, }\AttributeTok{mu =}\NormalTok{ mu0, }\AttributeTok{conf.level =} \DecValTok{1} \SpecialCharTok{{-}}\NormalTok{ alpha)}
\NormalTok{    vzorky }\OtherTok{\textless{}\textless{}{-}} \FunctionTok{bind\_rows}\NormalTok{(vzorky, }\FunctionTok{tibble}\NormalTok{(}\AttributeTok{n =}\NormalTok{ i, }\AttributeTok{vysledek =}\NormalTok{ test}\SpecialCharTok{$}\NormalTok{p.value }\SpecialCharTok{\textless{}=}\NormalTok{ alpha))}
\NormalTok{\}}
\NormalTok{vzorky}\SpecialCharTok{$}\NormalTok{cvysledky }\OtherTok{\textless{}{-}} \FunctionTok{cummean}\NormalTok{(vzorky}\SpecialCharTok{$}\NormalTok{vysledek)}
\FunctionTok{ggplot}\NormalTok{(vzorky, }\FunctionTok{aes}\NormalTok{(}\AttributeTok{x =}\NormalTok{ n, }\AttributeTok{y =}\NormalTok{ cvysledky)) }\SpecialCharTok{+}
    \FunctionTok{geom\_line}\NormalTok{() }\SpecialCharTok{+}
    \FunctionTok{geom\_hline}\NormalTok{(}\FunctionTok{aes}\NormalTok{(}\AttributeTok{yintercept =}\NormalTok{ .}\DecValTok{05}\NormalTok{, }\AttributeTok{color =} \StringTok{"red"}\NormalTok{), }\AttributeTok{linetype =} \StringTok{"dashed"}\NormalTok{) }\SpecialCharTok{+}
    \FunctionTok{scale\_y\_continuous}\NormalTok{(}\AttributeTok{limits =} \FunctionTok{c}\NormalTok{(}\DecValTok{0}\NormalTok{, .}\DecValTok{2}\NormalTok{)) }\SpecialCharTok{+}
    \FunctionTok{scale\_x\_continuous}\NormalTok{(}\AttributeTok{labels =}\NormalTok{ scales}\SpecialCharTok{::}\FunctionTok{label\_number}\NormalTok{()) }\SpecialCharTok{+}
    \FunctionTok{theme\_bw}\NormalTok{() }\SpecialCharTok{+}
    \FunctionTok{labs}\NormalTok{(}
        \AttributeTok{title =} \StringTok{"Procento falešných zamítnutí h0 se blíží hladině významnosti"}\NormalTok{,}
        \AttributeTok{y =} \StringTok{"Chyba I. druhu"}\NormalTok{,}
        \AttributeTok{x =} \StringTok{"Počet simulací"}
\NormalTok{    )}
\end{Highlighting}
\end{Shaded}

\includegraphics{index_files/figure-pdf/unnamed-chunk-2-1.pdf}

\section{Balíčky pro frekventistickou
inferenci}\label{baluxedux10dky-pro-frekventistickou-inferenci}

\subsection{base r}\label{base-r}

Test

test

\texttt{stats::t.test()}

\textbf{test}

Jednoduchý T-test

\begin{Shaded}
\begin{Highlighting}[]
\FunctionTok{set.seed}\NormalTok{(}\DecValTok{78}\NormalTok{)}
\NormalTok{x }\OtherTok{\textless{}{-}} \FunctionTok{rnorm}\NormalTok{(}\AttributeTok{n =} \DecValTok{100}\NormalTok{, }\AttributeTok{mean =} \DecValTok{1}\NormalTok{, }\AttributeTok{sd =} \DecValTok{1}\NormalTok{)}
\FunctionTok{t.test}\NormalTok{(}\AttributeTok{x =}\NormalTok{ x, }\AttributeTok{mu =} \DecValTok{0}\NormalTok{,}
       \AttributeTok{alternative =} \StringTok{"two.sided"}\NormalTok{, }\AttributeTok{conf.level =} \FloatTok{0.95}\NormalTok{)}
\end{Highlighting}
\end{Shaded}

\begin{verbatim}

    One Sample t-test

data:  x
t = 8.8438, df = 99, p-value = 3.621e-14
alternative hypothesis: true mean is not equal to 0
95 percent confidence interval:
 0.7158758 1.1300282
sample estimates:
mean of x 
 0.922952 
\end{verbatim}

Simulace alfa = chyba 1. druhu

\begin{Shaded}
\begin{Highlighting}[]
\FunctionTok{library}\NormalTok{(dplyr)}
\FunctionTok{library}\NormalTok{(ggplot2)}
\CommentTok{\# Nastavení}
\NormalTok{K   }\OtherTok{\textless{}{-}} \DecValTok{1000}   \CommentTok{\# Počet simulací}
\NormalTok{n   }\OtherTok{\textless{}{-}} \DecValTok{200}    \CommentTok{\# Velikost vzorku}
\NormalTok{mu0 }\OtherTok{\textless{}{-}} \DecValTok{0}      \CommentTok{\# Skutečný průměr}
\NormalTok{sd0 }\OtherTok{\textless{}{-}} \DecValTok{2}      \CommentTok{\# Populační směrodatná odchylka}
\NormalTok{alpha }\OtherTok{\textless{}{-}} \FloatTok{0.05} \CommentTok{\# Hladina významnosti}
\FunctionTok{set.seed}\NormalTok{(}\DecValTok{639}\NormalTok{)}
\NormalTok{vzorky }\OtherTok{\textless{}{-}} \FunctionTok{tibble}\NormalTok{()}

\CommentTok{\# Simulace}
\ControlFlowTok{for}\NormalTok{ (i }\ControlFlowTok{in} \FunctionTok{seq\_len}\NormalTok{(K)) \{}
\NormalTok{    x }\OtherTok{\textless{}{-}} \FunctionTok{rnorm}\NormalTok{(}\AttributeTok{n =}\NormalTok{ n, }\AttributeTok{mean =}\NormalTok{ mu0, }\AttributeTok{sd =}\NormalTok{ sd0)}
\NormalTok{    test }\OtherTok{\textless{}{-}} \FunctionTok{t.test}\NormalTok{(}\AttributeTok{x =}\NormalTok{ x, }\AttributeTok{mu =}\NormalTok{ mu0, }\AttributeTok{conf.level =} \DecValTok{1} \SpecialCharTok{{-}}\NormalTok{ alpha)}
\NormalTok{    vzorky }\OtherTok{\textless{}\textless{}{-}} \FunctionTok{bind\_rows}\NormalTok{(vzorky, }\FunctionTok{tibble}\NormalTok{(}\AttributeTok{n =}\NormalTok{ i, }\AttributeTok{vysledek =}\NormalTok{ test}\SpecialCharTok{$}\NormalTok{p.value }\SpecialCharTok{\textless{}=}\NormalTok{ alpha))}
\NormalTok{\}}
\NormalTok{vzorky}\SpecialCharTok{$}\NormalTok{cvysledky }\OtherTok{\textless{}{-}} \FunctionTok{cummean}\NormalTok{(vzorky}\SpecialCharTok{$}\NormalTok{vysledek)}
\FunctionTok{ggplot}\NormalTok{(vzorky, }\FunctionTok{aes}\NormalTok{(}\AttributeTok{x =}\NormalTok{ n, }\AttributeTok{y =}\NormalTok{ cvysledky)) }\SpecialCharTok{+}
    \FunctionTok{geom\_line}\NormalTok{() }\SpecialCharTok{+}
    \FunctionTok{geom\_hline}\NormalTok{(}\FunctionTok{aes}\NormalTok{(}\AttributeTok{yintercept =}\NormalTok{ .}\DecValTok{05}\NormalTok{, }\AttributeTok{color =} \StringTok{"red"}\NormalTok{), }\AttributeTok{linetype =} \StringTok{"dashed"}\NormalTok{) }\SpecialCharTok{+}
    \FunctionTok{scale\_y\_continuous}\NormalTok{(}\AttributeTok{limits =} \FunctionTok{c}\NormalTok{(}\DecValTok{0}\NormalTok{, .}\DecValTok{2}\NormalTok{)) }\SpecialCharTok{+}
    \FunctionTok{scale\_x\_continuous}\NormalTok{(}\AttributeTok{labels =}\NormalTok{ scales}\SpecialCharTok{::}\FunctionTok{label\_number}\NormalTok{()) }\SpecialCharTok{+}
    \FunctionTok{theme\_bw}\NormalTok{() }\SpecialCharTok{+}
    \FunctionTok{labs}\NormalTok{(}
        \AttributeTok{title =} \StringTok{"Procento falešných zamítnutí h0 se blíží hladině významnosti"}\NormalTok{,}
        \AttributeTok{y =} \StringTok{"Chyba I. druhu"}\NormalTok{,}
        \AttributeTok{x =} \StringTok{"Počet simulací"}
\NormalTok{    )}
\end{Highlighting}
\end{Shaded}

\includegraphics{index_files/figure-pdf/unnamed-chunk-4-1.pdf}

\subsection{easystats}\label{easystats}

použití, výhody/nevýhody

\subsection{inferencer}\label{inferencer}

použití, výhody/nevýhody

\section{Software pro bayesovskou
statistiku}\label{software-pro-bayesovskou-statistiku}

\subsection{Balíček R2WinBUGS}\label{baluxedux10dek-r2winbugs}

podporuje WinBUGS, OpenBUGS

\begin{Shaded}
\begin{Highlighting}[]
\FunctionTok{set.seed}\NormalTok{(}\DecValTok{123}\NormalTok{)}
\NormalTok{x }\OtherTok{\textless{}{-}} \FunctionTok{rbinom}\NormalTok{(}\DecValTok{10}\NormalTok{, }\DecValTok{1}\NormalTok{, .}\DecValTok{6}\NormalTok{)}

\NormalTok{bugs }\OtherTok{\textless{}{-}}\NormalTok{ R2WinBUGS}\SpecialCharTok{::}\FunctionTok{bugs}\NormalTok{(}
    \AttributeTok{data =} \FunctionTok{list}\NormalTok{(}
        \AttributeTok{N     =} \FunctionTok{length}\NormalTok{(x), }\CommentTok{\# Počet pozorování}
        \AttributeTok{x     =}\NormalTok{ x,         }\CommentTok{\# Vstupní data}
        \AttributeTok{alpha =} \FloatTok{0.01}\NormalTok{,      }\CommentTok{\# Hodnota parametru alpha}
        \AttributeTok{beta  =} \FloatTok{0.01}       \CommentTok{\# Hodnota parametru beta}
\NormalTok{    ),}
    \CommentTok{\# Počáteční hodnoty}
    \AttributeTok{inits =} \FunctionTok{list}\NormalTok{(}
        \FunctionTok{list}\NormalTok{(}\AttributeTok{p =} \FloatTok{0.5}\NormalTok{),}
        \FunctionTok{list}\NormalTok{(}\AttributeTok{p =} \FloatTok{0.5}\NormalTok{)}
\NormalTok{    ),}
    \AttributeTok{n.chains =} \DecValTok{2}\NormalTok{, }\AttributeTok{n.iter =} \DecValTok{5000}\NormalTok{, }\AttributeTok{n.burnin =} \DecValTok{1000}\NormalTok{, }\AttributeTok{n.thin =} \DecValTok{1}\NormalTok{,}
    \CommentTok{\# Parametry, které uložit}
    \AttributeTok{parameters.to.save =} \FunctionTok{c}\NormalTok{(}\StringTok{"p"}\NormalTok{),}
    \CommentTok{\# Cesta k modelu}
    \AttributeTok{working.directory =} \StringTok{"prakticka"}\NormalTok{,}
    \AttributeTok{model.file =} \StringTok{"r2winbugs.txt"}\NormalTok{,}
    \CommentTok{\# Cesta k programu WinBUGS}
    \AttributeTok{bugs.directory =}\NormalTok{ r}\StringTok{"(C:\textbackslash{}Users\textbackslash{}Mike\textbackslash{}Downloads\textbackslash{}WinBUGS14\textbackslash{}WinBUGS14)"}\NormalTok{,}
    \CommentTok{\# Odstraň pracovní soubory}
    \AttributeTok{clearWD =}\NormalTok{ T,}
    \CommentTok{\# Replikovatelnost}
    \AttributeTok{bugs.seed =} \DecValTok{123}
\NormalTok{)}
\end{Highlighting}
\end{Shaded}

Výsledek

\begin{Shaded}
\begin{Highlighting}[]
\FunctionTok{print}\NormalTok{(bugs)}
\end{Highlighting}
\end{Shaded}

\begin{verbatim}
Inference for Bugs model at "r2winbugs.txt", fit using WinBUGS,
 2 chains, each with 5000 iterations (first 1000 discarded)
 n.sims = 8000 iterations saved
         mean  sd 2.5%  25%  50%  75% 97.5% Rhat n.eff
p         0.6 0.1  0.3  0.5  0.6  0.7   0.9    1  4500
deviance 14.5 1.5 13.5 13.6 13.9 14.8  18.8    1  8000

For each parameter, n.eff is a crude measure of effective sample size,
and Rhat is the potential scale reduction factor (at convergence, Rhat=1).

DIC info (using the rule, pD = Dbar-Dhat)
pD = 1.0 and DIC = 15.6
DIC is an estimate of expected predictive error (lower deviance is better).
\end{verbatim}

Odhad parametru p.

\begin{Shaded}
\begin{Highlighting}[]
\NormalTok{mcmcplots}\SpecialCharTok{::}\FunctionTok{caterplot}\NormalTok{(}\AttributeTok{mcmcout =}\NormalTok{ bugs,             }\CommentTok{\# Výstup modelu}
                     \AttributeTok{parms =} \StringTok{"p"}\NormalTok{,                }\CommentTok{\# Vybraný parametr}
                     \AttributeTok{val.lim =} \FunctionTok{c}\NormalTok{(}\DecValTok{0}\NormalTok{, }\DecValTok{1}\NormalTok{),          }\CommentTok{\# Limity na ose X}
                     \AttributeTok{quantiles =} \FunctionTok{list}\NormalTok{(}
                        \AttributeTok{outer =} \FunctionTok{c}\NormalTok{(}\FloatTok{0.025}\NormalTok{, }\FloatTok{0.975}\NormalTok{), }\CommentTok{\# 95\% interval kredibility}
                        \AttributeTok{inner =} \FunctionTok{c}\NormalTok{(}\FloatTok{0.055}\NormalTok{, }\FloatTok{0.945}\NormalTok{)  }\CommentTok{\# 89\% interval kredibility}
\NormalTok{                     )}
\NormalTok{)}
\end{Highlighting}
\end{Shaded}

\includegraphics{index_files/figure-pdf/unnamed-chunk-7-1.pdf}

Posteriorní rozdělení jednotlivých chainů.

\begin{Shaded}
\begin{Highlighting}[]
\NormalTok{mcmcplots}\SpecialCharTok{::}\FunctionTok{denplot}\NormalTok{(}\AttributeTok{mcmcout =}\NormalTok{ bugs, }\CommentTok{\# Výstup modelu}
                   \AttributeTok{parms =} \StringTok{"p"}\NormalTok{,    }\CommentTok{\# Vybraný parametr}
                   \AttributeTok{xlim =} \FunctionTok{c}\NormalTok{(}\DecValTok{0}\NormalTok{, }\DecValTok{1}\NormalTok{), }\CommentTok{\# Limity na ose X}
                   \AttributeTok{ci =} \FloatTok{0.89}       \CommentTok{\# 89\% Interval kredibility}
\NormalTok{)}
\end{Highlighting}
\end{Shaded}

\includegraphics{index_files/figure-pdf/unnamed-chunk-8-1.pdf}

Vývoj jednotlivých chainů.

\begin{Shaded}
\begin{Highlighting}[]
\NormalTok{mcmcplots}\SpecialCharTok{::}\FunctionTok{traplot}\NormalTok{(}\AttributeTok{mcmcout =}\NormalTok{ bugs, }\CommentTok{\# Výstup modelu}
                   \AttributeTok{parms =} \StringTok{"p"}\NormalTok{,    }\CommentTok{\# Vybraný parametr}
                   \AttributeTok{ylim =} \FunctionTok{c}\NormalTok{(}\DecValTok{0}\NormalTok{, }\DecValTok{1}\NormalTok{)  }\CommentTok{\# Limity na ose Y}
\NormalTok{)}
\end{Highlighting}
\end{Shaded}

\includegraphics{index_files/figure-pdf/unnamed-chunk-9-1.pdf}

\subsection{Balíček jags}\label{baluxedux10dek-jags}

aplikace, R implementace, výhody/nevýhody, používá gibse

\subsection{stan}\label{stan}

aplikace, R implementace, výhody/nevýhody, používá hmc

\section{Simulace}\label{simulace}

jak budou simulace provedné, jak budou vyhodnocené, nastavení
ROPE/alternativ. pro odhad chyb

\subsection{Malý vzorek}\label{maluxfd-vzorek}

Bayes vs.~vybraný vzorec vs.~binomic

\subsection{Velký vzorek}\label{velkuxfd-vzorek}

Bayes vs.~vybraný vzorec vs.~binomic

\subsection{Porovnání výsledků}\label{porovnuxe1nuxed-vuxfdsledkux16f}

Jak testy dopadly



\end{document}
