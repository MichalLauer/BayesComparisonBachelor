% Options for packages loaded elsewhere
\PassOptionsToPackage{unicode}{hyperref}
\PassOptionsToPackage{hyphens}{url}
\PassOptionsToPackage{dvipsnames,svgnames,x11names}{xcolor}
%
\documentclass[
  11pt,
  a4paper]{report}

\usepackage{amsmath,amssymb}
\usepackage{iftex}
\ifPDFTeX
  \usepackage[T1]{fontenc}
  \usepackage[utf8]{inputenc}
  \usepackage{textcomp} % provide euro and other symbols
\else % if luatex or xetex
  \usepackage{unicode-math}
  \defaultfontfeatures{Scale=MatchLowercase}
  \defaultfontfeatures[\rmfamily]{Ligatures=TeX,Scale=1}
\fi
\usepackage{lmodern}
\ifPDFTeX\else  
    % xetex/luatex font selection
\fi
% Use upquote if available, for straight quotes in verbatim environments
\IfFileExists{upquote.sty}{\usepackage{upquote}}{}
\IfFileExists{microtype.sty}{% use microtype if available
  \usepackage[]{microtype}
  \UseMicrotypeSet[protrusion]{basicmath} % disable protrusion for tt fonts
}{}
\makeatletter
\@ifundefined{KOMAClassName}{% if non-KOMA class
  \IfFileExists{parskip.sty}{%
    \usepackage{parskip}
  }{% else
    \setlength{\parindent}{0pt}
    \setlength{\parskip}{6pt plus 2pt minus 1pt}}
}{% if KOMA class
  \KOMAoptions{parskip=half}}
\makeatother
\usepackage{xcolor}
\setlength{\emergencystretch}{3em} % prevent overfull lines
\setcounter{secnumdepth}{-\maxdimen} % remove section numbering
% Make \paragraph and \subparagraph free-standing
\ifx\paragraph\undefined\else
  \let\oldparagraph\paragraph
  \renewcommand{\paragraph}[1]{\oldparagraph{#1}\mbox{}}
\fi
\ifx\subparagraph\undefined\else
  \let\oldsubparagraph\subparagraph
  \renewcommand{\subparagraph}[1]{\oldsubparagraph{#1}\mbox{}}
\fi


\providecommand{\tightlist}{%
  \setlength{\itemsep}{0pt}\setlength{\parskip}{0pt}}\usepackage{longtable,booktabs,array}
\usepackage{calc} % for calculating minipage widths
% Correct order of tables after \paragraph or \subparagraph
\usepackage{etoolbox}
\makeatletter
\patchcmd\longtable{\par}{\if@noskipsec\mbox{}\fi\par}{}{}
\makeatother
% Allow footnotes in longtable head/foot
\IfFileExists{footnotehyper.sty}{\usepackage{footnotehyper}}{\usepackage{footnote}}
\makesavenoteenv{longtable}
\usepackage{graphicx}
\makeatletter
\def\maxwidth{\ifdim\Gin@nat@width>\linewidth\linewidth\else\Gin@nat@width\fi}
\def\maxheight{\ifdim\Gin@nat@height>\textheight\textheight\else\Gin@nat@height\fi}
\makeatother
% Scale images if necessary, so that they will not overflow the page
% margins by default, and it is still possible to overwrite the defaults
% using explicit options in \includegraphics[width, height, ...]{}
\setkeys{Gin}{width=\maxwidth,height=\maxheight,keepaspectratio}
% Set default figure placement to htbp
\makeatletter
\def\fps@figure{htbp}
\makeatother

\usepackage[top=25mm,bottom=25mm,right=25mm,left=30mm,head=12.5mm,foot=12.5mm]{geometry}
\let\openright=\clearpage

%%% Pokud tiskneme oboustranně:
%\documentclass[11pt,a4paper,twoside,openright]{report}
%\usepackage[top=25mm,bottom=25mm,right=25mm,left=30mm,head=12.5mm,foot=12.5mm]{geometry}
%\let\openright=\cleardoublepage

%%% DEFINICE ZÁKLADNÍCH PROMĚNNÝCH
\def\TypPrace{BP}                % bakalářská práce/bachelor thesis
%\def\TypPrace{DP}               % diplomová práce/master thesis
\def\Jazyk{cze}                  % čeština/czech

%%% Definice různých užitečných maker (viz popis uvnitř souboru)
\input{./setup/makra}

%%% Název práce v jazyce práce (přesně podle zadání)
%%% Title of the thesis in the language used in the text (exact according to assignment)
\def\NazevPrace{Odhad relativní četnosti binomického rozdělení pomocí klasického a bayesovského přístupu v jazyce R}

%%% Jméno autora
%%% Author's name - First name Surname
\def\AutorPrace{[Bc. Michal Lauer]}

%%% Rok odevzdání a měsíc (slovně)
%%% Year of submission and month (verbally) - month YYYY
\def\DatumOdevzdani{Prosinec 2024}

%%% Vedoucí práce: Jméno a příjmení s~tituly
%%% Supervisor: First name and surname with titles
\def\Vedouci{[Ing. Ondřej Vilikus, Ph.D.]}

% %%% Konzultant práce: Jméno a příjmení s~tituly
% %%% Consultant: First name and surname with titles
\def\Konzultant{}

%%% Studijní program
%%% Study program
\def\StudijniProgram{[Data Analytics]}

%%% Studijní program - specializace
%%% Study program - specialization
\def\Specializace{}

%%% Nepovinné poděkování (vedoucímu práce, konzultantovi, tomu, kdo zapůjčil software, literaturu apod.)
%%% Optional thanks (the supervisor, the consultant, the borrower of software, literature, etc.)
\def\Podekovani{%
Děkuji svému vedoucímu za odborné vedení práce a průběžné konzultace a své přítelkyni za neocenitelnou podporu.
}

%%% Abstrakt (doporučený rozsah cca 150-250 slov; nejedná se o zadání práce)
\def\Abstrakt{%
Abstrakt.
}
\def\AbstraktEN{%
Abstract.
}

%%% 3 až 5 klíčových slov (doporučeno)
\def\KlicovaSlova{Bayesovská statistika, odhad relativní četnosti, jazyk R}
\def\KlicovaSlovaEN{Bayesian statistics, relative frequency estimation, R language}

% Quarto fixes
\setcounter{secnumdepth}{2}
\makeatletter
\@ifpackageloaded{caption}{}{\usepackage{caption}}
\AtBeginDocument{%
\ifdefined\contentsname
  \renewcommand*\contentsname{Obsah}
\else
  \newcommand\contentsname{Obsah}
\fi
\ifdefined\listfigurename
  \renewcommand*\listfigurename{Seznam obrázků}
\else
  \newcommand\listfigurename{Seznam obrázků}
\fi
\ifdefined\listtablename
  \renewcommand*\listtablename{Seznam tabulek}
\else
  \newcommand\listtablename{Seznam tabulek}
\fi
\ifdefined\figurename
  \renewcommand*\figurename{Obrázek}
\else
  \newcommand\figurename{Obrázek}
\fi
\ifdefined\tablename
  \renewcommand*\tablename{Tabulka}
\else
  \newcommand\tablename{Tabulka}
\fi
}
\@ifpackageloaded{float}{}{\usepackage{float}}
\floatstyle{ruled}
\@ifundefined{c@chapter}{\newfloat{codelisting}{h}{lop}}{\newfloat{codelisting}{h}{lop}[chapter]}
\floatname{codelisting}{Výpis}
\newcommand*\listoflistings{\listof{codelisting}{Seznam výpisů}}
\makeatother
\makeatletter
\makeatother
\makeatletter
\@ifpackageloaded{caption}{}{\usepackage{caption}}
\@ifpackageloaded{subcaption}{}{\usepackage{subcaption}}
\makeatother
\ifLuaTeX
\usepackage[bidi=basic]{babel}
\else
\usepackage[bidi=default]{babel}
\fi
\babelprovide[main,import]{czech}
% get rid of language-specific shorthands (see #6817):
\let\LanguageShortHands\languageshorthands
\def\languageshorthands#1{}
\ifLuaTeX
  \usepackage{selnolig}  % disable illegal ligatures
\fi
\usepackage{bookmark}

\IfFileExists{xurl.sty}{\usepackage{xurl}}{} % add URL line breaks if available
\urlstyle{same} % disable monospaced font for URLs
\hypersetup{
  pdflang={cs},
  colorlinks=true,
  linkcolor={blue},
  filecolor={Maroon},
  citecolor={Blue},
  urlcolor={Blue},
  pdfcreator={LaTeX via pandoc}}

\author{}
\date{}

\begin{document}

\include{interim/intro.tex}

\include{interim/toc.tex}

\openright
\listoffigures
Poznámka: Seznam obrázků je vhodné použít, pokud počet obrázků v textu práce je větší než 20.

\clearpage
\listoftables
Poznámka: Seznam tabulek je vhodný použít, pokud počet tabulek v textu práce je větší než 20.

\clearpage
\lstlistoflistings
Poznámka: Seznam výpisů programového kódu je vhodné použít, pokud počet vložených objektů tohoto typu je větší než 20. 

\include{interim/zkratky.tex}

\include{uvod.tex}

\chapter{Statistické metody}\label{statistickuxe9-metody}

Krátký úvod do historie, bayes, inferenční bayes (rozdělení)
vs.~inference (bod) citace Karla

\section{Inference}\label{inference}

proč to používáme, výběr vs.~populace, reprezentativnost

\subsection{Problematika výběrových
šetření}\label{problematika-vuxfdbux11brovuxfdch-ux161etux159enuxed}

reprezentativnost, definice populace, čas sběru, organizace
sběru\ldots{}

\section{Frekventistická inference}\label{frekventistickuxe1-inference}

Jak to funguje, jak to spoléhá na sampling distributions

\subsection{Testování hypotéz}\label{testovuxe1nuxed-hypotuxe9z}

hladina významnosti, úroveň spolehlivosti, Testovací statistika,
kritický obor, 1/2 stranný test p-hodnota, interval spolehlivosti

\subsection{Metriky při testování
hypotéz}\label{metriky-pux159i-testovuxe1nuxed-hypotuxe9z}

Chyba I. a II. druhu, síla testu, velikost efektu

\subsection{Jednovýběrový odhad poměru s velkým
vzorkem}\label{jednovuxfdbux11brovuxfd-odhad-pomux11bru-s-velkuxfdm-vzorkem}

použití, předpoklady, poměrový Z test, binomický test, síla testu,
velikost efektu

\subsection{Jednovýběrový odhad poměru s malým
vzorkem}\label{jednovuxfdbux11brovuxfd-odhad-pomux11bru-s-maluxfdm-vzorkem}

Proč jsou důležité speciální metody, nějaké typy (wiki)

\section{Bayesovská inference}\label{bayesovskuxe1-inference}

Odvození bayesova vzorce, popis likelihood/aprior/data, druhy
aprior/posterior

\chapter{Monte Carlo generování}\label{monte-carlo-generovuxe1nuxed}

Halsing, Gibs, HMC

\section{Vyhodnocení generovaného
rozdělení}\label{vyhodnocenuxed-generovanuxe9ho-rozdux11blenuxed}

korelace, ESS, monte carlo error\ldots{}

\subsection{Vyhodocení hypotéz}\label{vyhodocenuxed-hypotuxe9z}

Interval kredibility, ROPE, Bayesův faktor

\subsection{Odhad poměru}\label{odhad-pomux11bru}

\chapter{Praktické odhady}\label{praktickuxe9-odhady}

\section{Balíčky pro frekventistickou
inferenci}\label{baluxedux10dky-pro-frekventistickou-inferenci}

\subsection{Klasické test poměru}\label{klasickuxe9-test-pomux11bru}

Test

test

\texttt{stats::t.test()}

\textbf{test}

Jednoduchý T-test

\begin{code}{R}{Moje caption}{zpracovani-python}
set.seed(78)
x <- rnorm(n = 100, mean = 1, sd = 1)
t.test(x = x, mu = 0,
       alternative = "two.sided", conf.level = 0.95)
\end{code}

\begin{verbatim}

    One Sample t-test

data:  x
t = 8.8438, df = 99, p-value = 3.621e-14
alternative hypothesis: true mean is not equal to 0
95 percent confidence interval:
 0.7158758 1.1300282
sample estimates:
mean of x 
 0.922952 
\end{verbatim}

\ref{zpracovani-python}

Simulace alfa = chyba 1. druhu

\begin{code}{R}{NENÍ CAPTION}{NENÍ LABEL}
library(dplyr)
\end{code}

\begin{code}{R}{NENÍ CAPTION}{NENÍ LABEL}
library(ggplot2)
# Nastavení
K   <- 1000   # Pocet simulaci
n   <- 200    # Velikost vzorku
mu0 <- 0      # Skutecny prumer
sd0 <- 2      # Populační směrodatná odchylka
alpha <- 0.05 # Hladina významnosti
set.seed(639)
vzorky <- tibble()

# Simulace
for (i in seq_len(K)) {
    x <- rnorm(n = n, mean = mu0, sd = sd0)
    test <- t.test(x = x, mu = mu0, conf.level = 1 - alpha)
    vzorky <<- bind_rows(vzorky, tibble(n = i, vysledek = test$p.value <= alpha))
}
vzorky$cvysledky <- cummean(vzorky$vysledek)
ggplot(vzorky, aes(x = n, y = cvysledky)) +
    geom_line() +
    geom_hline(aes(yintercept = .05, color = "red"), linetype = "dashed") +
    scale_y_continuous(limits = c(0, .2)) +
    scale_x_continuous(labels = scales::label_number()) +
    theme_bw() +
    labs(
        title = "Procento falešných zamítnutí h0 se blíží hladině významnosti",
        y = "Chyba I. druhu",
        x = "Počet simulací"
    )
\end{code}

\includegraphics{index_files/figure-pdf/unnamed-chunk-4-1.pdf}

\section{Software pro bayesovskou
statistiku}\label{software-pro-bayesovskou-statistiku}

\subsection{Balíček R2WinBUGS}\label{baluxedux10dek-r2winbugs}

podporuje WinBUGS, OpenBUGS

\begin{code}{R}{NENÍ CAPTION}{NENÍ LABEL}
set.seed(123)
x <- rbinom(10, 1, .6)

bugs <- R2WinBUGS::bugs(
    data = list(
        N     = length(x), # Počet pozorování
        x     = x,         # Vstupní data
        alpha = 0.01,      # Hodnota parametru alpha
        beta  = 0.01       # Hodnota parametru beta
    ),
    # Počáteční hodnoty
    inits = list(
        list(p = 0.5),
        list(p = 0.5)
    ),
    n.chains = 2, n.iter = 5000, n.burnin = 1000, n.thin = 1,
    # Parametry, které uložit
    parameters.to.save = c("p"),
    # Cesta k modelu
    working.directory = "prakticka",
    model.file = "r2winbugs.txt",
    # Cesta k programu WinBUGS
    bugs.directory = r"(C:\Users\Mike\Downloads\WinBUGS14\WinBUGS14)",
    # Odstraň pracovní soubory
    clearWD = T,
    # Replikovatelnost
    bugs.seed = 123
)
\end{code}

Výsledek

\begin{code}{R}{NENÍ CAPTION}{NENÍ LABEL}
print(bugs)
\end{code}

\begin{verbatim}
Inference for Bugs model at "r2winbugs.txt", fit using WinBUGS,
 2 chains, each with 5000 iterations (first 1000 discarded)
 n.sims = 8000 iterations saved
         mean  sd 2.5%  25%  50%  75% 97.5% Rhat n.eff
p         0.6 0.1  0.3  0.5  0.6  0.7   0.9    1  4500
deviance 14.5 1.5 13.5 13.6 13.9 14.8  18.8    1  8000

For each parameter, n.eff is a crude measure of effective sample size,
and Rhat is the potential scale reduction factor (at convergence, Rhat=1).

DIC info (using the rule, pD = Dbar-Dhat)
pD = 1.0 and DIC = 15.6
DIC is an estimate of expected predictive error (lower deviance is better).
\end{verbatim}

Odhad parametru p.

\begin{code}{R}{NENÍ CAPTION}{NENÍ LABEL}
mcmcplots::caterplot(mcmcout = bugs,             # Výstup modelu
                     parms = "p",                # Vybraný parametr
                     val.lim = c(0, 1),          # Limity na ose X
                     quantiles = list(
                        outer = c(0.025, 0.975), # 95% interval kredibility
                        inner = c(0.055, 0.945)  # 89% interval kredibility
                     )
)
\end{code}

\includegraphics{index_files/figure-pdf/unnamed-chunk-6-1.pdf}

Posteriorní rozdělení jednotlivých chainů.

\begin{code}{R}{NENÍ CAPTION}{NENÍ LABEL}
mcmcplots::denplot(mcmcout = bugs, # Výstup modelu
                   parms = "p",    # Vybraný parametr
                   xlim = c(0, 1), # Limity na ose X
                   ci = 0.89       # 89% Interval kredibility
)
\end{code}

\includegraphics{index_files/figure-pdf/unnamed-chunk-7-1.pdf}

Vývoj jednotlivých chainů.

\begin{code}{R}{NENÍ CAPTION}{NENÍ LABEL}
mcmcplots::traplot(mcmcout = bugs, # Výstup modelu
                   parms = "p",    # Vybraný parametr
                   ylim = c(0, 1)  # Limity na ose Y
)
\end{code}

\includegraphics{index_files/figure-pdf/unnamed-chunk-8-1.pdf}

\subsection{Balíček jags}\label{baluxedux10dek-jags}

aplikace, R implementace, výhody/nevýhody, používá gibse

\subsection{stan}\label{stan}

aplikace, R implementace, výhody/nevýhody, používá hmc

\section{Simulace}\label{simulace}

jak budou simulace provedné, jak budou vyhodnocené, nastavení
ROPE/alternativ. pro odhad chyb

\subsection{Malý vzorek}\label{maluxfd-vzorek}

Bayes vs.~vybraný vzorec vs.~binomic

\subsection{Velký vzorek}\label{velkuxfd-vzorek}

Bayes vs.~vybraný vzorec vs.~binomic

\subsection{Porovnání výsledků}\label{porovnuxe1nuxed-vuxfdsledkux16f}

Jak testy dopadly

\part*{\Prilohy\thispagestyle{empty}}
\appendix

\chapter{Bayesovské modely}\label{bayesovskuxe9-modely}

\begin{code}{R}{NENÍ CAPTION}{NENÍ LABEL}
print(1 + 1)
\end{code}

\begin{verbatim}
[1] 2
\end{verbatim}



\end{document}
