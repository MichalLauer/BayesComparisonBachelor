% Options for packages loaded elsewhere
\PassOptionsToPackage{unicode}{hyperref}
\PassOptionsToPackage{hyphens}{url}
\PassOptionsToPackage{dvipsnames,svgnames,x11names}{xcolor}
%
\documentclass[
  11pt,
  a4paper]{report}

\usepackage{amsmath,amssymb}
\usepackage{iftex}
\ifPDFTeX
  \usepackage[T1]{fontenc}
  \usepackage[utf8]{inputenc}
  \usepackage{textcomp} % provide euro and other symbols
\else % if luatex or xetex
  \usepackage{unicode-math}
  \defaultfontfeatures{Scale=MatchLowercase}
  \defaultfontfeatures[\rmfamily]{Ligatures=TeX,Scale=1}
\fi
\usepackage{lmodern}
\ifPDFTeX\else  
    % xetex/luatex font selection
\fi
% Use upquote if available, for straight quotes in verbatim environments
\IfFileExists{upquote.sty}{\usepackage{upquote}}{}
\IfFileExists{microtype.sty}{% use microtype if available
  \usepackage[]{microtype}
  \UseMicrotypeSet[protrusion]{basicmath} % disable protrusion for tt fonts
}{}
\makeatletter
\@ifundefined{KOMAClassName}{% if non-KOMA class
  \IfFileExists{parskip.sty}{%
    \usepackage{parskip}
  }{% else
    \setlength{\parindent}{0pt}
    \setlength{\parskip}{6pt plus 2pt minus 1pt}}
}{% if KOMA class
  \KOMAoptions{parskip=half}}
\makeatother
\usepackage{xcolor}
\setlength{\emergencystretch}{3em} % prevent overfull lines
\setcounter{secnumdepth}{-\maxdimen} % remove section numbering
% Make \paragraph and \subparagraph free-standing
\ifx\paragraph\undefined\else
  \let\oldparagraph\paragraph
  \renewcommand{\paragraph}[1]{\oldparagraph{#1}\mbox{}}
\fi
\ifx\subparagraph\undefined\else
  \let\oldsubparagraph\subparagraph
  \renewcommand{\subparagraph}[1]{\oldsubparagraph{#1}\mbox{}}
\fi

\usepackage{color}
\usepackage{fancyvrb}
\newcommand{\VerbBar}{|}
\newcommand{\VERB}{\Verb[commandchars=\\\{\}]}
\DefineVerbatimEnvironment{Highlighting}{Verbatim}{commandchars=\\\{\}}
% Add ',fontsize=\small' for more characters per line
\usepackage{framed}
\definecolor{shadecolor}{RGB}{241,243,245}
\newenvironment{Shaded}{\begin{snugshade}}{\end{snugshade}}
\newcommand{\AlertTok}[1]{\textcolor[rgb]{0.68,0.00,0.00}{#1}}
\newcommand{\AnnotationTok}[1]{\textcolor[rgb]{0.37,0.37,0.37}{#1}}
\newcommand{\AttributeTok}[1]{\textcolor[rgb]{0.40,0.45,0.13}{#1}}
\newcommand{\BaseNTok}[1]{\textcolor[rgb]{0.68,0.00,0.00}{#1}}
\newcommand{\BuiltInTok}[1]{\textcolor[rgb]{0.00,0.23,0.31}{#1}}
\newcommand{\CharTok}[1]{\textcolor[rgb]{0.13,0.47,0.30}{#1}}
\newcommand{\CommentTok}[1]{\textcolor[rgb]{0.37,0.37,0.37}{#1}}
\newcommand{\CommentVarTok}[1]{\textcolor[rgb]{0.37,0.37,0.37}{\textit{#1}}}
\newcommand{\ConstantTok}[1]{\textcolor[rgb]{0.56,0.35,0.01}{#1}}
\newcommand{\ControlFlowTok}[1]{\textcolor[rgb]{0.00,0.23,0.31}{#1}}
\newcommand{\DataTypeTok}[1]{\textcolor[rgb]{0.68,0.00,0.00}{#1}}
\newcommand{\DecValTok}[1]{\textcolor[rgb]{0.68,0.00,0.00}{#1}}
\newcommand{\DocumentationTok}[1]{\textcolor[rgb]{0.37,0.37,0.37}{\textit{#1}}}
\newcommand{\ErrorTok}[1]{\textcolor[rgb]{0.68,0.00,0.00}{#1}}
\newcommand{\ExtensionTok}[1]{\textcolor[rgb]{0.00,0.23,0.31}{#1}}
\newcommand{\FloatTok}[1]{\textcolor[rgb]{0.68,0.00,0.00}{#1}}
\newcommand{\FunctionTok}[1]{\textcolor[rgb]{0.28,0.35,0.67}{#1}}
\newcommand{\ImportTok}[1]{\textcolor[rgb]{0.00,0.46,0.62}{#1}}
\newcommand{\InformationTok}[1]{\textcolor[rgb]{0.37,0.37,0.37}{#1}}
\newcommand{\KeywordTok}[1]{\textcolor[rgb]{0.00,0.23,0.31}{#1}}
\newcommand{\NormalTok}[1]{\textcolor[rgb]{0.00,0.23,0.31}{#1}}
\newcommand{\OperatorTok}[1]{\textcolor[rgb]{0.37,0.37,0.37}{#1}}
\newcommand{\OtherTok}[1]{\textcolor[rgb]{0.00,0.23,0.31}{#1}}
\newcommand{\PreprocessorTok}[1]{\textcolor[rgb]{0.68,0.00,0.00}{#1}}
\newcommand{\RegionMarkerTok}[1]{\textcolor[rgb]{0.00,0.23,0.31}{#1}}
\newcommand{\SpecialCharTok}[1]{\textcolor[rgb]{0.37,0.37,0.37}{#1}}
\newcommand{\SpecialStringTok}[1]{\textcolor[rgb]{0.13,0.47,0.30}{#1}}
\newcommand{\StringTok}[1]{\textcolor[rgb]{0.13,0.47,0.30}{#1}}
\newcommand{\VariableTok}[1]{\textcolor[rgb]{0.07,0.07,0.07}{#1}}
\newcommand{\VerbatimStringTok}[1]{\textcolor[rgb]{0.13,0.47,0.30}{#1}}
\newcommand{\WarningTok}[1]{\textcolor[rgb]{0.37,0.37,0.37}{\textit{#1}}}

\providecommand{\tightlist}{%
  \setlength{\itemsep}{0pt}\setlength{\parskip}{0pt}}\usepackage{longtable,booktabs,array}
\usepackage{calc} % for calculating minipage widths
% Correct order of tables after \paragraph or \subparagraph
\usepackage{etoolbox}
\makeatletter
\patchcmd\longtable{\par}{\if@noskipsec\mbox{}\fi\par}{}{}
\makeatother
% Allow footnotes in longtable head/foot
\IfFileExists{footnotehyper.sty}{\usepackage{footnotehyper}}{\usepackage{footnote}}
\makesavenoteenv{longtable}
\usepackage{graphicx}
\makeatletter
\def\maxwidth{\ifdim\Gin@nat@width>\linewidth\linewidth\else\Gin@nat@width\fi}
\def\maxheight{\ifdim\Gin@nat@height>\textheight\textheight\else\Gin@nat@height\fi}
\makeatother
% Scale images if necessary, so that they will not overflow the page
% margins by default, and it is still possible to overwrite the defaults
% using explicit options in \includegraphics[width, height, ...]{}
\setkeys{Gin}{width=\maxwidth,height=\maxheight,keepaspectratio}
% Set default figure placement to htbp
\makeatletter
\def\fps@figure{htbp}
\makeatother

\usepackage[top=25mm,bottom=25mm,right=25mm,left=30mm,head=12.5mm,foot=12.5mm]{geometry}
\let\openright=\clearpage

%%% Pokud tiskneme oboustranně:
%\documentclass[11pt,a4paper,twoside,openright]{report}
%\usepackage[top=25mm,bottom=25mm,right=25mm,left=30mm,head=12.5mm,foot=12.5mm]{geometry}
%\let\openright=\cleardoublepage

%%% DEFINICE ZÁKLADNÍCH PROMĚNNÝCH
\def\TypPrace{BP}                % bakalářská práce/bachelor thesis
%\def\TypPrace{DP}               % diplomová práce/master thesis
\def\Jazyk{cze}                  % čeština/czech

%%% Definice různých užitečných maker (viz popis uvnitř souboru)
\input{./setup/makra}

%%% Název práce v jazyce práce (přesně podle zadání)
%%% Title of the thesis in the language used in the text (exact according to assignment)
\def\NazevPrace{Odhad relativní četnosti binomického rozdělení pomocí klasického a bayesovského přístupu v jazyce R}

%%% Jméno autora
%%% Author's name - First name Surname
\def\AutorPrace{[Bc. Michal Lauer]}

%%% Rok odevzdání a měsíc (slovně)
%%% Year of submission and month (verbally) - month YYYY
\def\DatumOdevzdani{Prosinec 2024}

%%% Vedoucí práce: Jméno a příjmení s~tituly
%%% Supervisor: First name and surname with titles
\def\Vedouci{[Ing. Ondřej Vilikus, Ph.D.]}

% %%% Konzultant práce: Jméno a příjmení s~tituly
% %%% Consultant: First name and surname with titles
\def\Konzultant{}

%%% Studijní program
%%% Study program
\def\StudijniProgram{[Data Analytics]}

%%% Studijní program - specializace
%%% Study program - specialization
\def\Specializace{}

%%% Nepovinné poděkování (vedoucímu práce, konzultantovi, tomu, kdo zapůjčil software, literaturu apod.)
%%% Optional thanks (the supervisor, the consultant, the borrower of software, literature, etc.)
\def\Podekovani{%
Děkuji svému vedoucímu za odborné vedení práce a průběžné konzultace a své přítelkyni za neocenitelnou podporu.
}

%%% Abstrakt (doporučený rozsah cca 150-250 slov; nejedná se o zadání práce)
\def\Abstrakt{%
Abstrakt.
}
\def\AbstraktEN{%
Abstract.
}

%%% 3 až 5 klíčových slov (doporučeno)
\def\KlicovaSlova{Bayesovská statistika, odhad relativní četnosti, jazyk R}
\def\KlicovaSlovaEN{Bayesian statistics, relative frequency estimation, R language}

% Quarto fixes
\setcounter{secnumdepth}{2}
\makeatletter
\@ifpackageloaded{caption}{}{\usepackage{caption}}
\AtBeginDocument{%
\ifdefined\contentsname
  \renewcommand*\contentsname{Obsah}
\else
  \newcommand\contentsname{Obsah}
\fi
\ifdefined\listfigurename
  \renewcommand*\listfigurename{Seznam obrázků}
\else
  \newcommand\listfigurename{Seznam obrázků}
\fi
\ifdefined\listtablename
  \renewcommand*\listtablename{Seznam tabulek}
\else
  \newcommand\listtablename{Seznam tabulek}
\fi
\ifdefined\figurename
  \renewcommand*\figurename{Obrázek}
\else
  \newcommand\figurename{Obrázek}
\fi
\ifdefined\tablename
  \renewcommand*\tablename{Tabulka}
\else
  \newcommand\tablename{Tabulka}
\fi
}
\@ifpackageloaded{float}{}{\usepackage{float}}
\floatstyle{ruled}
\@ifundefined{c@chapter}{\newfloat{codelisting}{h}{lop}}{\newfloat{codelisting}{h}{lop}[chapter]}
\floatname{codelisting}{Výpis}
\newcommand*\listoflistings{\listof{codelisting}{Seznam výpisů}}
\makeatother
\makeatletter
\makeatother
\makeatletter
\@ifpackageloaded{caption}{}{\usepackage{caption}}
\@ifpackageloaded{subcaption}{}{\usepackage{subcaption}}
\makeatother
\ifLuaTeX
\usepackage[bidi=basic]{babel}
\else
\usepackage[bidi=default]{babel}
\fi
\babelprovide[main,import]{czech}
% get rid of language-specific shorthands (see #6817):
\let\LanguageShortHands\languageshorthands
\def\languageshorthands#1{}
\ifLuaTeX
  \usepackage{selnolig}  % disable illegal ligatures
\fi
\usepackage{bookmark}

\IfFileExists{xurl.sty}{\usepackage{xurl}}{} % add URL line breaks if available
\urlstyle{same} % disable monospaced font for URLs
\hypersetup{
  pdflang={cs},
  colorlinks=true,
  linkcolor={blue},
  filecolor={Maroon},
  citecolor={Blue},
  urlcolor={Blue},
  pdfcreator={LaTeX via pandoc}}

\author{}
\date{}

\begin{document}

%%% Intro
\pagestyle{empty}
\hypersetup{pageanchor=false}

\begin{center}
\Huge\sffamily
\VSE\\
\FIS

\vspace{\stretch{1}}

\includegraphics[width=.5\textwidth]{img/FIS_2_logo_2_rgb_CZ}

\vspace{\stretch{2}}

\bfseries\NazevPrace

\vspace{8mm}
\mdseries\TypPraceText

\vspace{8mm}
\large
\begin{tabular}{rl}
\StudijniProgramText: & \StudijniProgram \\
\ifthenelse{\equal{\Specializace}{}}{%
    % empty value
    }{
    \rule{0pt}{6mm}%
    \SpecializaceText: & \Specializace \\
}
\end{tabular}

\vspace{\stretch{8}}

\begin{tabular}{rl}
\AutorText: & \AutorPrace \\
\noalign{\vspace{2mm}}
\VedouciText: & \Vedouci \\
\ifthenelse{\equal{\Konzultant}{}}{%
    % empty value
    }{
    \rule{0pt}{6mm}%
    \KonzultantText: & \Konzultant \\
}
\end{tabular}

\vspace{8mm}
\Praha, \DatumOdevzdani
\end{center}


%%% Poděkování
\hypersetup{pageanchor=true}
\cleardoublepage
\pagestyle{plain}
\openright
\vspace*{\fill}
\section*{\PodekovaniText}
\noindent
\Podekovani
\vspace{1cm}


%%% Povinná informační strana práce
\openright
\section*{Abstrakt}
\noindent
\Abstrakt
\subsection*{Klíčová slova}
\noindent
\KlicovaSlova

\bigskip\bigskip\bigskip\bigskip\bigskip
\section*{Abstract}
\noindent
\AbstraktEN
\subsection*{Keywords}
\noindent
\KlicovaSlovaEN

\openright

%%% Obsah
\setcounter{tocdepth}{2}
\tableofcontents

%%% Seznam obrázků
\openright
\listoffigures

%%% Seznam tabulek
\clearpage
\listoftables

%%% Seznam kódu
\clearpage
\lstlistoflistings

%%% Zkratky
\chapter*{\SeznamZkratek}

\begin{multicols}{2}
\raggedright
\begin{description}
\item [BUGS] Bayesian inference Using Gibbs Sampling
\end{description}
\end{multicols}

\chapter*{Úvod}
\addcontentsline{toc}{chapter}{Úvod}

Tohle je \textbf{úvodní} \emph{text}.

\chapter{Statistické metody}\label{statistickuxe9-metody}

Krátký úvod do historie, bayes, inferenční bayes (rozdělení)
vs.~inference (bod) citace Karla

\section{Inference}\label{inference}

proč to používáme, výběr vs.~populace, reprezentativnost

\subsection{Problematika výběrových
šetření}\label{problematika-vuxfdbux11brovuxfdch-ux161etux159enuxed}

reprezentativnost, definice populace, čas sběru, organizace
sběru\ldots{}

\section{Frekventistická inference}\label{frekventistickuxe1-inference}

Jak to funguje, jak to spoléhá na sampling distributions

\subsection{Testování hypotéz}\label{testovuxe1nuxed-hypotuxe9z}

hladina významnosti, úroveň spolehlivosti, Testovací statistika,
kritický obor, 1/2 stranný test p-hodnota, interval spolehlivosti

\subsection{Metriky při testování
hypotéz}\label{metriky-pux159i-testovuxe1nuxed-hypotuxe9z}

Chyba I. a II. druhu, síla testu, velikost efektu

\subsection{Jednovýběrový odhad poměru s velkým
vzorkem}\label{jednovuxfdbux11brovuxfd-odhad-pomux11bru-s-velkuxfdm-vzorkem}

použití, předpoklady, poměrový Z test, binomický test, síla testu,
velikost efektu

\subsection{Jednovýběrový odhad poměru s malým
vzorkem}\label{jednovuxfdbux11brovuxfd-odhad-pomux11bru-s-maluxfdm-vzorkem}

Proč jsou důležité speciální metody, nějaké typy (wiki)

\section{Bayesovská inference}\label{bayesovskuxe1-inference}

Odvození bayesova vzorce, popis likelihood/aprior/data, druhy
aprior/posterior

\chapter{Monte Carlo generování}\label{monte-carlo-generovuxe1nuxed}

Halsing, Gibs, HMC

\section{Vyhodnocení generovaného
rozdělení}\label{vyhodnocenuxed-generovanuxe9ho-rozdux11blenuxed}

korelace, ESS, monte carlo error\ldots{}

\subsection{Vyhodocení hypotéz}\label{vyhodocenuxed-hypotuxe9z}

Interval kredibility, ROPE, Bayesův faktor

\subsection{Odhad poměru}\label{odhad-pomux11bru}

\chapter{Praktické odhady}\label{praktickuxe9-odhady}

\section{Balíčky pro frekventistickou
inferenci}\label{baluxedux10dky-pro-frekventistickou-inferenci}

\subsection{Klasické test poměru}\label{klasickuxe9-test-pomux11bru}

Test

test

\texttt{stats::t.test()}

\textbf{test}

Jednoduchý T-test

\ref{zpracovani-python}

Simulace alfa = chyba 1. druhu

\section{Software pro bayesovskou
statistiku}\label{software-pro-bayesovskou-statistiku}

\subsection{Software BUGS}\label{software-bugs}

První software, který se snažil zpopularizovat bayesovské metody je
software BUGS (\emph{Bayesian inference Using Gibbs Sampling}). Projekt,
jehož cílem bylo vytvořit software pro bayesovskou statistiku, započal
už v roce 1989 na oddělní biostatistiky na univerzitě Cambridge a vedl k
vytvoření programu BUGS. Postupem času se program vyvynul do nástroje
WinBUGS, který je prezentován v této práci
\parencite{TheBayesianScientificWorkingGroup2024_BayesianScientificWork}.

Prví verze programu byla představena na Bayesovské konferenci ve
Valencii v roce 1991 a později byla distribuovaná na disketách. Z
počátku byly implementovány s Gibsovým vzorkováním pouze jendnoduché
metody a úpravy, jako adaptivní vzorkování nebo bayesovské inverze. Po
čase byla implementována i velmi limitující verze
Metropolisova-Hastingova algoritmu, která fungovala na bázi mřížek
\footnote{Jedná se o tzn. \emph{grid-search}, kdy se hledají správné
  hodnoty v předem definované tabulce.}.

V 90. letech 20. století se projekt přesunul do Imperial College v
Londýně a program BUGS se začal vyvíjet i pro osobní počítače se
systémem Windows. Do verze WinBUGS se časem implementovala komplexní a
plnohodnotná verze Metropolisova-Hestingova algoritmu, která dokázala
pracovat bez aproximační tabulky. Velkou sílou však bylo, že uživatel
dokázal definovat své apriorní předpoklady, data, vztahy a cykly v
grafickém prostředí. Díky tomu mohli program WinBUGS používat i lidé bez
zkušeností v programování.

V roce 2004 započala na univerzitě v Helsinách práce na projektu s
názvem OpenBUGS, který měl za úkol splnit tři primární cíle:

\begin{enumerate}
\def\labelenumi{\arabic{enumi})}
\tightlist
\item
  rozdělit funkčnost softwaru od jejího vzhledu,
\item
  udělat verzi nezávislou na operačním prostředí, a
\item
  vytvožit experimentálnější prostředí pro zkoušení nových metod.
\end{enumerate}

Pro splnění prvního cíle byl vyvynut nástroj s názevm BRugs. Díky
oddělení funkčnosti a vzhledu lze BRugs napojit na další programy jako
SAS, Excel nebo R a využívat tak simulací BUGS i mimo jeho prostředí. Ke
splnění druhého cíle byla vyvynutá další verze softwaru bugs s názvem
LinBUGS pro vývoj v linuxovém prostředí procesory Intel. Experimentální
prostředí pro testování nových metod byla vytvořena open-source
\footnote{software, který má veřejný zdrojový kód; tzn. \emph{otevřený
  software}.} verze OpenBUGS. Vývoj obou aplikací se časem rozdělil a
obě mají silné stránky v něčem jiném. Software OpenBUGS dokáže
flexibilně měnit simulační metody nebo simulace provádět paralelně.
Vývoj softwaru WinBUGS mířil primárně na excelenci v epidemiologickém a
farmakokinetikou.

Aplikace z rodiny BUGS však trpěli několika nedostatky. Tím hlavním je,
že jejich vývoj závisí na hrstce vývojářu a i přes to, že je kód
(alespoň pro software OpenBUGS) veřejně dostupný, je vývoj
problematický. To je způsobeno primárně komplexitou kódu a programovacím
jazykem, ve kterém je software napsaný. Druhou velkou nevýhodou je právě
programovací jazyk, který se nedokáže vyrovnat např. jazyku C++, ve
kterém jsou napsána většina novodobých simulačních programů
\parencite{LunnEtAl2009_BUGSProjectEvolution}.

V programovacím jazyce R lze s programy BUGS komunikovat pomocí hned
několika balíčků. První možností jsou balíčku R2WinBUGS a R2OpenBUGS
\parencite{SturtzEtAl2005_R2WinBUGSPackageRunning}, které umí
komunikovat s programy WinBUGS a OpenBUGS. Jelikož byl balíček OpenBUGS
aktualizován naposledy v r. 2020, je smysluplnější\footnote{aktuální k
  27. Sprnu, 2024.} v případě zájmu používat balíček R2WinBUGS, který
poslední aktualizaci obdržel v r. 2024. Balíčky nenabízejí aplikační
přístup, ale slouží jako automatizační pomůcka. Pokud funkcím předáte
data, model a parametry simulací, tak se spustí zvolený software
(OpenBUGS nebo WinBUGS), do kterého jsou automaticky zvoleny preference
uživatel pomocí skriptu. Po dokončení se program sám zavře a výsledky
jsou dopstupné v jazyce R.

Druhou možností je balíček BRugs
\parencite{ThomasEtAl2006_MakingBUGSOpen}, který dokáže komunikovat s
OpenBUGS pomocí programatického prostředí a není tedy nutné, aby
balíček/funkce na pozadí spouštěli samotné programy. Na rozdíl od
balíčku R2WinBUGS, který externě spouští program, je BRugs více
flexibilní. Nastavení probíhá pomocí metod, které nastavují, spouští a
inicializují jednotlivé řetěze a simulace. Nevýhodou balíčku je, že je
nutné ho kompilovat, což může být pro méně zkušené uživatele náročné.

\subsubsection{Balíček \{R2WinBUGS\}}\label{baluxedux10dek-r2winbugs}

Výsledek

Odhad parametru p.

Posteriorní rozdělení jednotlivých chainů.

Vývoj jednotlivých chainů.

Autokorelace.

\subsection{Balíček JAGS}\label{baluxedux10dek-jags}

2003 https://www.jstor.org/stable/26447820

JAGS - Just Another Gibbs Sampler

\subsubsection{Balíček \{rjags\}}\label{baluxedux10dek-rjags}

Tvorba modelu

Adaptační doba, která se volá automaticky.

Burn-in generování, je to pro každý chainu.

Generování vzorků z každého chainu.

Základní plot

\subsubsection{Balíček \{R2jags\}}\label{baluxedux10dek-r2jags}

Možná úplně vynechat???

Divně spojený bugs and jags.

\begin{itemize}
\tightlist
\item
  Lze komplikovaně nastavit stejný seed
\end{itemize}

for (i in 1:n.chains) \{ init.values{[}{[}i{]}{]} \textless-
inits{[}{[}i{]}{]} init.values{[}{[}i{]}{]}\(.RNG.name <- RNGname
    init.values[[i]]\).RNG.seed \textless- runif(1, 0, 2\^{}31) \}

(asi by to šlo nastavit seed a pak to generovat setjně pomocí runif i
nahoře)

\begin{itemize}
\tightlist
\item
  adapt = burnin nebo adapt = 100
\end{itemize}

if (n.burnin \textgreater{} 0) \{ n.adapt \textless- n.burnin \} else \{
n.adapt \textless- 100 \}

\begin{itemize}
\tightlist
\item
  Lze paralelizovat pomocí jags.parallel
\end{itemize}

Výsledky jsou pořád ze stejného posteriorního rozdělení a jsou validní,
akorát se charakteristiky nerovnají.

\subsubsection{Visualizace}\label{visualizace}

Catterplot.

Posteriorní rozdělení.

Trace plot.

Autokorelace.

\subsection{stan}\label{stan}

Stan 1.0, 2012

http://www.stat.columbia.edu/\textasciitilde gelman/research/published/stan\_jebs\_2.pdf

Stan (není delší název).

aplikace, R implementace, výhody/nevýhody, používá hmc

\subsubsection{Balíček \{rstan\}}\label{baluxedux10dek-rstan}

Model.

\begin{Shaded}
\begin{Highlighting}[]
\FunctionTok{set.seed}\NormalTok{(}\DecValTok{26}\NormalTok{)}
\NormalTok{x }\OtherTok{\textless{}{-}} \FunctionTok{rbinom}\NormalTok{(}\DecValTok{10}\NormalTok{, }\DecValTok{1}\NormalTok{, }\FloatTok{0.6}\NormalTok{)}

\CommentTok{\# stan.model \textless{}{-} rstan::stan(}
\CommentTok{\#     model\_code = readLines("./kapitoly/modely/stan.txt"),}
\CommentTok{\#     model\_name = "Jednaoduchý příklad",}
\CommentTok{\#     data = list(}
\CommentTok{\#         N     = length(x), \# Počet pozorování}
\CommentTok{\#         x     = x,         \# Vstupní data}
\CommentTok{\#         alpha = 0.1,       \# Hodnota parametru alpha}
\CommentTok{\#         beta  = 0.1        \# Hodnota parametru beta}
\CommentTok{\#     ),}
\CommentTok{\#     init = list(}
\CommentTok{\#         list(p = 0.5),}
\CommentTok{\#         list(p = 0.5)}
\CommentTok{\#     ),}
\CommentTok{\#     chains = 2,}
\CommentTok{\#     iter = 2000 + 5000,}
\CommentTok{\#     warmup = 2000,}
\CommentTok{\#     thin = 1,}
\CommentTok{\#     seed = 59}
\CommentTok{\# )}
\end{Highlighting}
\end{Shaded}

\section{Simulace}\label{simulace}

jak budou simulace provedné, jak budou vyhodnocené, nastavení
ROPE/alternativ. pro odhad chyb

\subsection{Malý vzorek}\label{maluxfd-vzorek}

Bayes vs.~vybraný vzorec vs.~binomic

\subsection{Velký vzorek}\label{velkuxfd-vzorek}

Bayes vs.~vybraný vzorec vs.~binomic

\subsection{Porovnání výsledků}\label{porovnuxe1nuxed-vuxfdsledkux16f}

Jak testy dopadly

%%% Závěr
{
\pagestyle{plain}
\chapter*{Závěr}
\addcontentsline{toc}{chapter}{Závěr}

Konec práce, závěr.

}

%%% Literatura
\printbibliography[keyword={literature},title={\bibnamex},heading={bibintoc}]
\printbibliography[keyword={package},title={\bibnamey},heading={bibintoc}]

%%% Přílohy
\part*{\Prilohy\thispagestyle{empty}}
\appendix

\chapter{Bayesovské modely}\label{bayesovskuxe9-modely}



\end{document}
